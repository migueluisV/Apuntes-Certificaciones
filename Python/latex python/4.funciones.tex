\section{Funciones}

Una \textbf{función} es un bloque de código que puede ser llamado dentro de otra sección del código para realizar una tarea en específico. En todo lenguaje de programación, las funciones están constituidas de un \textbf{nombre} y \textbf{argumentos}, como hemos visto a lo largo de estos apuntes. Podemos crear nuestras propias funciones con la palabra reservada \textbf{def}, seguido del nombre y los argumentos, como vemos en la siguiente estructura:
\begin{lstlisting}
    def nombre_funcion(argumentos):
	# Código.
\end{lstlisting}

Como vemos, sigue la misma estructura que la declaración de un ciclo o una condicional, dentro de alguna otra función, mandamos a llamar a nuestra función creada escribiendo su nombre y sus argumentos (en caso de que tenga, sino, solo escribimos el par de paréntesis), debe recordar que la función debe ser declarada previamente a su llamada:
\begin{lstlisting}
    # Función que recibe el nombre del usuario y lo despliega.
    def Nombre():
	nombre = str(input("Escribe tu nombre"))
	apellidos = str(input("Escribe tus apellidos: "))
	print(nombre + " " + apellidos)

    # Función que despliega un mensaje y un nombre.
    def principal():
	print("Hola, mi nombre es ")
	Nombre() # Llamada a la función creada.
\end{lstlisting}


\subsection{Funciones que regresan valor}

Las funciones que no regresan valor (las que fueron definidas en la sección anterior) les podemos llamar \textbf{procedimientos}, que son funciones que no regresan valor, aquellos bloques de código repetibles y reusables que regresan algún tipo de valor les denominaremos funciones \textbf{funciones}, para tener clara una diferencia entre el tipo que regresa y el que no un valor. Para que una función regrese un valor, se declara como ya vimos, y utilizamos la palabra reservada \textbf{return}, seguido del valor que queremos que se regrese, este puede ser entero, cadena, decimal o boleano.
\begin{lstlisting}
    def Potencia(x, y): # Función que regresa un número elevado a una potencia.
	return x * y # Regresa la potenciación.

    # Ingreso de valores a las variables.
    a = int(input("Ingresa un número: "))
    b = int(input("Ingresa otro número: "))

    # Imprime el resultado.
    print("El número " + a + " elevado a la " + b + " es: "Cuadrado(a, b))
\end{lstlisting}

Como vemos, la función declarada utiliza dos argumentos en su declaración, además de que la llamamos dentro de print, por lo que puede actuar como si fuera una variable, podemos utilizar esta función para asignarle un valor a una variable si tuviéramos la necesidad.


\subsection{Funciones Lambda}

Existen dos formas de crear funciones en Python: declarándolas con la palabra \textbf{def}, con un nombre, argumentos y conjunto de instrucciones, y creando la función con la palabra reservada \textbf{lambda}. Podemos decir que las funciones lambda son hechas en el camino, para un uso rápido, y que contiene solo una instrucción, su estructura consiste en la palabra lambda, los parámetros que se van a utilizar, dos puntos, la instrucción a realizar, y el valor de o los parámetros, como vemos enseguida:
\begin{lstlisting}
    # Función con def.
    def polynomial(x):
        return x**2 + 5*x + 4
    print(polynomial(-4))

    # Función con lambda.
    print((lambda x: x**2 + 5*x + 4) (-4))
\end{lstlisting}

Entonces, vemos que solo se puede realizar una instrucción con este tipo de instrucciones, que también son llamadas \textbf{funciones anónimas}.


\subsection{Funciones map y filter}

La función \textbf{map} regresa un objeto iterable, y recibe como parámetros una función y otro objeto iterable (que puede ser un conjunto de datos o ciclos), podemos decir que la función map repite, en un objeto iterable, una función que interactúe con un objeto iterable, es como decir que a cada elemento de una lista se le realice una acción, como sumar cinco. Veamos como se utiliza:
\begin{lstlisting}
    # Declara una lista con números.
    numeros = [3, 4, 5, 6, 7, 8]

    # Se utiliza map para multiplica cada uno de los elementos de la lista por dos con una
    # función lambda, el resultado se convierte en lista.
    lista = list(map(lambda x: x * 2, numeros))

    print(lista) # Imprime [6, 8, 10, 12, 14, 16]
\end{lstlisting}

La función \textbf{filter} se encarga de tomar únicamente ciertos elementos de un objeto iterable que cumplan con una condición (llamada \textbf{predicado}):
\begin{lstlisting}
    # Declara una lista con números.
    numeros = [3, 4, 5, 6, 7, 8]
    
    # Se utiliza filter para obtener los elementos mayores a cinco con una
    # función lambda, el resultado se convierte en lista.
    lista = list(filter(lambda x: x>5, numeros))

    print(lista) # Imprime [6, 7, 8].
\end{lstlisting}


\subsection{Generadores}

Los \textbf{generadores} son un tipo iterable, como listas o tuplas, que no poseen alguna forma de acceder a los elementos que lo componen, pero que pueden ser recorridos con ciclos. Este tipo de iterable se utiliza dentro de funciones que serán repetidas en ciclos, las variables o acciones que se puedan borrar entre iteración e iteración se mantendrán intactas gracias a palabra reservada \textbf{yield}, que es como el return de la función, esta palabra es lo que define a una función como generadora.
\begin{lstlisting}
    def numeros():
        x = 5
        while x > 0:
            yield x
            x -= 1

    for x in numeros():
        print(x)
\end{lstlisting}

Debemos recordar poner alguna condición que acabe con el ciclo o el generador, para que no se repitan las instrucciones de manera infinita.


\subsection{Decoradores}

Los \textbf{decoradores} son funciones que pueden modificar otras funciones, de tal forma que la decoración o modificación sea algo leve, pero despliega de otra forma los resultados de la función a decorar, para lograr este cometido, debemos declarar una función dentro de otra función, como vemos a continuación:
\begin{lstlisting}
    # Declara función para decorar otra función como parámetro.
    def decor(fun):
        # Función que realiza la modificación a la función parámetro.
        def partir():
            # Agrega iguales al despliegue de resultados.
            print("================")
            # Llamada a la función parámetro.
            fun()
            print("================")
        # Regresa la función que realizó la modificación.
        return partir

    # Función que será modificada.
    def imprimir_texto():
        print("Hola mundo")

    # Variable que recibe la función modificada.
    decoracion = decor(imprimir_texto)
    # Llamada a la función modificada.
    decoracion()

    #Despliega:
    #================
    #Hola mundo
    #================
\end{lstlisting}

La función \textit{decor} recibe una función como parámetro (llamada \textit{fun}), dentro, se declara la función \textit{partir} (declaración de función dentro de otra), dentro de la función partir, se realiza la decoración o modificación a la función que se quiere decorar, vemos que en el ejemplo anterior se agregan iguales encima y debajo de la función parámetro \textit{fun}, finalmente, se regresa a la propia función \textit{partir}, vemos que tenemos una función a decorar llamada \textit{imprimir\_ texto}, que simplemente imprime "Hola mundo", la variable \textit{decoracion} recibe el resultado de llamar a \textit{decor} con \textit{imprimir\_ texto} como parámetro, ya para terminar, la variable se convierte en un tipo de función, por lo que le agregamos paréntesis al final para poder ver el resultado.

Utilizamos una variable para almacenar la función decorada, sin embargo, podemos hacer que la función a modificar reciba a ella misma, pero con la modificación ya hecha, además, podemos utilizar el arroba (\@) antes de la función a modificar, seguido del nombre de la función decoradora (\textit{decor} en el ejemplo anterior). Veamos el ejemplo anterior pero con lo que acabamos de mencionar:
\begin{lstlisting}
    # Declara función para decorar otra función como parámetro.
    def decor(fun):
        # Función que realiza la modificación a la función parámetro.
        def partir():
            # Agrega iguales al despliegue de resultados.
            print("================")
            # Llamada a la función parámetro.
            fun()
            print("================")
        # Regresa la función que realizó la modificación.
        return partir

    # Funcion que sera modificada.
    @decor
    def imprimir_texto():
        print("Hola mundo")

    #La función a modificar será modificada en el momento.
    imprimir_texto = decor(imprimir_texto)
    # Llamada a la función modificada.
    imprimir_texto()   

    # Despliega:
    #================
    #Hola mundo
    #================
\end{lstlisting}


\subsection{Parámetros *args y **kwargs}

Podemos hacer que una función reciba una cantidad indeterminada de parámetros con \textbf{*args}, es decir, en una llamada puede recibir dos parámetros, en otra llamada, puede recibir cuatro, en otra, solamente tres, la forma en la que se reciben estos parámetros es por medio de \textbf{tuplas}. Veamos el siguiente ejemplo:
\begin{lstlisting}
    # Define función que recibe una cantidad variable de parámetros con *args.
    def funcion(*args):
        # Imprime el/los parámetros recibidos.
        print(args)

    # Llamada a la función y se le mandan distintos numeros de parámetros.   
    funcion(1, 2, 3, 4, 5)  # Imprime (1, 2, 3, 4 5).
    funcion(1, 2)			# Imprime (1, 2).
    funcion(2, 3, 4)		# Imprime (2, 3, 4).
    funcion(4)				# Imprime (4, ).
\end{lstlisting}

Como vemos, si llamamos a la misma función con distinta cantidad de parámetros y los imprimimos, se reciben e imprimen correctamente, como este tipo de parámetro es una tupla, podemos acceder a los elementos de la misma con el operador [], podemos hacer que la función tenga, por ejemplo, tres parámetros, donde dos de ellos no posean el asterisco y uno si, como vemos enseguida.
\begin{lstlisting}
    # Define función que recibe dos parámetros y tupla variable de parámetros con *args.
    def funcion(par1, par2, *args):
        # Imprime el/los parámetros recibidos.
        print(args)
        print(args[0])
        print(par1)
        print(par2)

    # Llamada a la función y se le mandan distintos numeros de parámetros.   
    funcion(1, 2, 3, 4, 5)  # Imprime (3, 4, 5) 3 1 2.
    funcion(1, 2)			# Imprime () 1 2.
    funcion(2, 3, 4)		# Imprime (4, ) 2 3.
\end{lstlisting}

Entonces, podemos tener tuplas variables de parámetros y parámetros fijos, en caso de que la función tenga tupla de parámetros y no se reciban en la llamada a la función, se despliegan una tupla vacía. El parámetro \textbf{*kwargs} (por las palabras \textit{keywords arguments}) permite recibir parámetros como si fueran variables, es decir, una variable con su nombre y su tipo de dato, la función trata a estos parámetros como listas, donde el nombre de la variable es la llave, y el valor de la variable es el valor de la llave; veamos un ejemplo para que quede más claro:
\begin{lstlisting}
    # Define función que recibe dos parámetro fijos, una tupla de parámetros y un diccionario de parámetros.
    def my_func(x, y=7, *args, **kwargs):
        print(x)		# Imprime el parámetro x.
        print(y)		# Imprime el parámetro y, que inicialmente vale 7.
        print(args)     # Imprime una tupla de parámetros.
        print(kwargs)	# Imprime un diccionario de parámetros.

    # Llamada a la función con distintos tipos de parámetros.
    my_func(2, 3, 4, 5, t = 1, a=7, b=8)
    # Imprime:
    # 2
    # 3
    # (4, 5)
    # { 't':1, 'a':7, 'b':8 }
\end{lstlisting}

Al igual que con las tuplas de parámetros, para acceder a los elementos de los diccionarios parámetros utilizamos el nombre del parámetro y el operador [], donde aparece el nombre de la llave a buscar o utilizar.


\subsection{Listas como parámetros}

No es más que agregar un nombre de variable a los parámetros de una función, y tratar a este nombre como una lista dentro de la función, como vemos en el siguiente ejemplo:
\begin{lstlisting}
    # Función que regresa la suma de todos los elementos de una lista.
    mi_funcion(lista):
	# Declaración de variable.
	suma=0
	# Ciclo que recorre los elementos de la lista parámetro.
        for x in range(len(lista)):
            suma=suma+lista[x]
   	return suma
\end{lstlisting}


\subsection{Sobrecarga de métodos}

En otros lenguajes, como en CSharp, Java o C++, podemos crear una función o método con un nombre y cierto número de parámetros, posterior a esto, podemos crear nuevamente la función, pero con una cantidad distinta de parámetros, y no necesariamente la función debe regresar el mismo tipo de dato, esto es la \textbf{sobrecarga de métodos}; la llamada de la función será dependiendo de la cantidad de parámetros que le pasemos.

Sin embargo, esto no existe directamente en Python, se puede simular utilizando los parámetros *args, veamos el siguiente ejemplo:
\begin{lstlisting}
    # Regresa el perímetro de un triángulo.
    def perimetro(*args):
        if args[0] != 0:
            # Perímetro de un triángulo equilátero.
            return args[0] * 3
        elif args[1] != 0:
            # Perímetro de un triángulo isósceles.
            return (args[1] * 2) + args[2]
        elif args[3] != 0:
            # Perímetro de un triángulo escaleno.
            return args[3] + args[4] + args[5]
\end{lstlisting}

Vemos que la función tiene el tipo de parámetro especial donde podemos mandarle una cantidad de argumentos en una llamada y otra cantidad distinta en futuras llamadas; la función evalúa si el primer elemento de la tupla del parámetro es distinto de 0, si es así, regresa un valor, esto aplica para la primer función sobrecargada con \textit{n} cantidad de parámetros (en este caso: 1); si es falta esta evaluación, evalúa si el segundo elemento de la tupla es distinto de 0, si es así, regresa el perímetro de un triángulo isósceles, el cual requiere de dos valores (altura y base) para ser calculado, lo cual correspondería con otra versión de la función sobrecargada donde se reciben dos parámetros en lugar de uno; por último, si la segunda evaluación es falsa, verifica si el cuarto elemento en la tupla es distinto de 0, si es cierto, regresa el perímetro de un triángulo escaleno, el cual requiere de tres valores (lado 1, lado 2 y lado 3) para ser calculado, coincidiendo así con una tercera versión de la función sobrecargada.

Lo que describimos anteriormente sería lo mismo al siguiente ejemplo en CSharp:
\begin{lstlisting}
    // Método Perimetro.
    public float Perimetro(float lado)
    {
        return 3 * lado;
    }

    // Método sobrecargado una vez.
    public float Perimetro(float altura, float base)
    {
        return (2 * altura) + base;
    }

    // Método sobrecargado dos veces.
    public float Perimetro(float lado1, float lado2, float lado3)
    {
        return lado1 + lado2 + lado3;
    }
\end{lstlisting}

Como vemos, CSharp requiere de tres declaraciones y de la misma función con distinta cantidad de parámetros, Python simula esto con los parámetros *args pero, al poder recibir en una llamada un argumento, en la siguiente dos, en la siguiente tres, debemos especificar dentro de la función qué comportamiento debe adoptar en cualquier de los casos que les acabamos de describir, es por ello que la función evalúa si el primer, segundo y cuarto elemento de la tupla es distinto de 0.


\subsection{Números aleatorios}

Es necesario importar el módulo \textbf{random} para poder generar números aleatorios:
\begin{lstlisting}
    import random
\end{lstlisting}

La función \textbf{random()} te genera un número decimal entre 0.0 y 1.0 (excluye a estos dos):
\begin{lstlisting}
    import random

    for i in range(5):
        print(random.random())

    # Imprime:
    # 0.6355590910913725.
    # 0.38870490605141683.
    # 0.3757381647176976.
    # 0.38770694501889935.
    # 0.22472067642268556.
\end{lstlisting}


\subsubsection{Números aleatorios enteros entre dos valores}

La función \textbf{randint(x, y)} te genera un número entero entre dos valores (incluidos):
\begin{lstlisting}
    import random

    for i in range(5):
        print(random.randint(1, 10))

    # Imprime:
    # 6.
    # 3.
    # 7.
    # 3.
    # 2.
\end{lstlisting}


\subsubsection{Números aleatorios enteros entre dos valores con salto}

La función \textbf{randrange(x, y, salto)} te genera un número entero entre dos valores (incluidos) dando un salto a partir del valor mínimo ingresado:
\begin{lstlisting}
    import random

    for i in range(5):
        print(random.randrange(5, 27, 4))

    # Imprime:
    # 17.
    # 13.
    # 5.
    # 17.
    # 13.
\end{lstlisting}


\subsubsection{Números aleatorios decimales entre dos valores}

La función \textbf{uniform(x, y)} te genera un número decimal entre dos valores (incluidos):
\begin{lstlisting}
    import random

    for i in range(5):
        print(random.uniform(100, 200))

    # Imprime:
    # 170.3543065193162.
    # 103.47025653056637.
    # 126.52588283656675.
    # 169.60671144065486.
    # 145.21872629322894.
\end{lstlisting}



\section{Recursividad}

La \textbf{recursividad} es que una función se llame a sí misma una cantidad finita de veces, teniendo que poseer un \textbf{caso} o \textbf{condición} para que las llamadas a la función acabe, sino, sería indeterminada; la recursividad es una alternativa al uso de ciclos, el ejemplo más clásico es el factorial:
\begin{lstlisting}
    # Declara función recursiva.
    def factorial(x):
        # Condición de salida de la recursividad.
        if x == 1:
            return 1
        else:
            # Regresa el parámetro multiplicado por la misma función con el parámetro alterado.
            return x * factorial(x-1)

    # Despliega 120.
    print(factorial(5))
\end{lstlisting}

El ejemplo anterior corresponde con la \textbf{recursividad directa}, la \textbf{recursividad indirecta} corresponde con la llamada de una función 1 a la función 2, y luego la función 2 llama a la función 1, así durante n cantidad de veces.



\section{Programación funcional}

Este es un estilo de programación que se enfoca en las \textbf{funciones}, es decir, programaremos alrededor de una función, que reciba un valor, y regrese otro, como las funciones matemáticas (f(x)). Las funciones que reciben otra función como parámetro o argumento y regresa un valor o función son llamadas \textbf{funciones de orden mayor}. Este tipo de programación no conlleva palabras reservadas, simplemente es un estilo de programar. Veamos el siguiente ejemplo para ver de qué estamos hablando:
\begin{lstlisting}
    # Declara función que se llama a sí misma.
    def doble(fun, para):
        return fun(fun(para))

    # Declara función que suma 5 a x.
    def cinco_mas(x):
        return x + 5
   
    # Imprime la función doble, le pasa como parámetro la función
    # cinco_mas, y el valor a sumar.
    print(doble(cinco_mas, 10)) #Imprime 20
\end{lstlisting}

El sentido aquí es que se llama a la función de orden mayor, esta recibe como parámetro otra función a ejecutar, más uno o más parámetros que serán manipulados por la función a ejecutar, dentro de la función de orden mayor, se regresa a ella misma, que a su vez se llama a ella misma, dándole sentido al nombre de la función, ejecuta la función que fue pasada originalmente como parámetro, junto con el parámetro que igual fue pasado al inicio, el resultado, si ejecuta este código en algún compilador de Python podrán analizar y comprender un poco más el funcionamiento de la programación funcional.


\subsection{Funciones puras e impuras}

Existen tipos de funciones en este estilo, nos encontramos con las \textbf{funciones puras}, las cuales reciben parámetros y los utilizan para regresar un valor independiente a estos parámetros, los cuales nunca son modificados; por otro lado, tenemos las \textbf{funciones impuras}, las cuales si modifican los parámetros que se reciben, podemos ver los siguientes ejemplos:
\begin{lstlisting}
    # Declara lista vacía.
   numeros = []

    # Declara función que recibe dos parámetros, asigna un calculo con los
    # parámetros a una variable y la regresa.
    def pura(x, y):
        temp = x * y
        return temp

    # Declara función que agrega el parámetro recibido a la lista vacía.
    def impura(num):
        numeros.append(num)
\end{lstlisting}

Así pues, la función "pura" calcula un valor para 'temp' con los parámetros recibidos, y se regresa la variable 'temp', convirtiéndola en pura por el simple hecho de no modificar los valores de los parámetros, por otro lado, la función "impura" si bien no modifica el valor del parámetro recibido, modifica una lista declarada anteriormente, volviéndola impura.

Las \textbf{ventajas} de las funciones puras es que son fáciles de entender, probar y son eficientes en rendimiento, la \textbf{desventaja} es que esta fácilidad para entender lo vuelve difícil de escribir en algunas situaciones.
