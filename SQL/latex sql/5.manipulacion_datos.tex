\section{Más funciones}

Para continuar con los siguientes temas, el nuevo aspecto de la tabla "Personas" aparece en la \textit{Tabla \ref{tab: 26}}:
\begin{table}[H]
    \centering
    \caption{Tabla "Personas" re acondicionada}
    \label{tab: 26}
    \begin{tabular}{|l|l|l|l|l|}
        \hline
        \textbf{id} & \textbf{nombre} & \textbf{apellidos} & \textbf{ciudad} & \textbf{edad} \\
        \hline
        1 & John        & Smith     & New York      & 24 \\
        \hline
        2 & David       & Williams  & Los Angeles   & 42 \\
        \hline
        3 & Chloe       & Anderson  & Chicago       & 65 \\
        \hline
        4 & Emily       & Adams     & Houston       & \textit{NULL} \\
        \hline
        5 & James       & Roberts   & Philadelphia  & 31 \\
        \hline
        6 & Andrew      & Thomas    & New York      & 21 \\
        \hline
        7 & Daniel      & Harris    & New York      & 67 \\
        \hline
        8 & Charlotte   & Walker    & Chicago       & \textit{NULL} \\
        \hline
        9 & Samuel      & Clark     & San Diego     & \textit{NULL} \\
        \hline
        10 & Anthony    & Young     & Los Angeles   & 52 \\
        \hline
    \end{tabular}
\end{table}


\subsection{De cadenas}

Vimos que SQL posee algunas funciones especiales que permiten realizar conteos u operaciones a las columnas y su contenido, la \textit{Tabla \ref{tab: 27}} contiene más funciones, enfocadas a las cadenas de texto, puede consultar el resto de funciones en este \href{https://www.tutorialspoint.com/sql/sql-string-functions.htm}{enlace}:
\begin{table}[H]
    \centering
    \caption{Funciones para cadenas}
    \label{tab: 27}
    \begin{tabular}{m{6cm} m{8cm}}
        \hline
        \textbf{Función} & \textbf{Descripción} \\
        \hline
        CONCAT(c1, c2, ..., c\textit{n})        & Une dos o más cadenas o datos de distintas columnas \\
        LOWER(columna)                          & Convierte todas las cadenas de una columna a minúsculas \\
        \parbox{5cm}{SUBSTRING(columna,\\comienzo, fin)}       & Extrae una subcadena de una cadena, es decir, de todos los datos de una columna \\
        \parbox{5cm}{REPLACE(columna,\\a-remplazar, remplazo)} & Remplaza una cadena por otra en todos los datos de una columna \\
        LENGTH(cadena)                          & Regresa el total de caracteres de una cadena \\
        FORMAT(numero, decimales)               & Regresa un número decimal con cierta cantidad de decimales \\
        \hline
    \end{tabular}
\end{table}

Puede estas funciones con las tablas mostradas en este documento para ver los resultados. Además, podemos utilizar funciones en conjunto:
\begin{lstlisting}
    SELECT CONCAT(
        SUBSTRING(nombre, 1, 1),
        '. ',
        UPPER(apellidos)) AS name
    FROM Personas
\end{lstlisting}

La sentencia anterior toma la primer letra de cada registro de la columna "nombre" y las combina con cada registro de la columna "apellidos", todo en mayúsculas, quedando el resultado de la siguiente manera (\textit{Tabla \ref{tab: 28}}):
\begin{table}[H]
    \centering
    \caption{Uso de varias funciones en una sola sentencia}
    \label{tab: 28}
    \begin{tabular}{|l|}
        \hline
        \textbf{name} \\
        \hline
        J. SMITH \\
        \hline
        D. WILLIAMS \\
        \hline
        C. ANDERSON \\
        \hline
        E. ADAMS \\
        \hline
        J. ROBERTS \\
        \hline
        A. THOMAS \\
        \hline
        D. HARRIS \\
        \hline
        C. WALKER \\
        \hline
        S. CLARK \\
        \hline
        A. YOUNG \\
        \hline
    \end{tabular}
\end{table}



\subsection{Agregación y matemáticas}

Ya hay punto en ese documento referente a las funciones matemáticas, la \textit{Tabla \ref{tab: 29}} contiene los operadores aritméticos permitidos en SQL, puede consultar más en este \href{https://www.w3schools.com/sql/sql_operators.asp}{enlace}:
\begin{table}[H]
    \centering
    \caption{Operadores aritméticos en SQL}
    \label{tab: 29}
    \begin{tabular}{l l}
        \hline
        \textbf{Función} & \textbf{Descripción} \\
        \hline
        +   & Suma \\
        -   & Resta \\
        $*$   & Multiplicación \\
        /   & División \\
        \%  & Módulo \\
        \hline
    \end{tabular}
\end{table}

Estos operadores pueden ser aplicados en sentencias para renombrar una columna de una cierta manera:
\begin{lstlisting}
    SELECT nombre, apellidos, edad / 2 AS menos
    FROM Personas
\end{lstlisting}

Esta sentencia creará una columna con el nombre, apellidos y la mitad de edad de cada persona registrada. Otro ejemplo sería el siguiente:
\begin{lstlisting}
    SELECT nombre, apellidos, peso / (altura * altura) AS imc
    FROM Personas
\end{lstlisting}

Suponiendo que en nuestra tabla "Personas" existen columnas llamadas "peso" y "altura" creamos una columna llamada "imc" con la división del peso entre el cuadrado de la altura, es decir, varias operaciones aritméticas en una sola sentencia.



\section{CASE}

Si recordamos de los lenguajes de programación, existen las estructuras condicional, como el \textit{if} y \textit{case}, SQL puede utilizar la estructura \textit{case} mediante el comando \textbf{CASE} para asignar valores a registros según un criterio o condición. Veamos el siguiente ejemplo y el resultado en la \textit{Tabla \ref{tab: 30}}:
\begin{lstlisting}
    SELECT nombre, apellidos,
    CASE
        WHEN edad >= 65 THEN 'Senior'
        WHEN edad >= 25 AND edad < 65 THEN 'Adulto'
        ELSE 'Joven'
    END AS categoria
    FROM Personas
\end{lstlisting}
\begin{table}[H]
    \centering
    \caption{Uso de CASE para asignación de valores}
    \label{tab: 30}
    \begin{tabular}{|l|l|l|}
        \hline
        \textbf{nombre} & \textbf{apellidos} & \textbf{categoria} \\
        \hline
        John        & Smith     & Joven \\
        \hline
        David       & Williams  & Adulto \\
        \hline
        Chloe       & Anderson  & Senior \\
        \hline
        Emily       & Adams     & Joven \\
        \hline
        James       & Roberts   & Adulto \\
        \hline
        Andrew      & Thomas    & Joven \\
        \hline
        Daniel      & Harris    & Senior \\
        \hline
        Charlotte   & Walker    & Joven \\
        \hline
        Samuel      & Clark     & Joven \\
        \hline
        Anthony     & Young     & Adulto \\
        \hline
    \end{tabular}
\end{table}

El ejemplo anterior crear una columna personalizada llamada "categoria" con ayuda del comando \textit{CASE}: el comando \textbf{WHEN} ayuda a poner la condición a seguir para asignar un valor, esta primera sub-sentencia asigna el valor "Senior" con el comando \textbf{THEN} a la columna si la edad es mayor igual a 65; la segunda sub-sentencia asigna el valor "Adulto" a la columna si la edad está entre 25 y 64, el comando \textbf{ELSE} asigna el valor "Joven" en caso de que las dos sub-sentencias anteriores no se cumplan (algo como el \textit{if-else}); no olvide agregar el comando \textbf{END} al final de su CASE para indicarle a SQL que ese es el final de la condicional. Podemos utilizar cuantos comando \textit{WHEN} deseemos y omitir el uso del comando \textit{ELSE}.
