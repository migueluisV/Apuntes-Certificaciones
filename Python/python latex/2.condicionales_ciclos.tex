\section{Condicionales}
\hspace{0.55cm}La estructura \textbf{if-else} permite ejecutar un bloque de código en caso de que su expresión se cumpla, en caso contrario, puede ejecutarse otro bloque de código. La estructura de declaración es:
\begin{lstlisting}
	if (condicion):
		#Instrucciones.
	else:
		#Instrucciones.
\end{lstlisting}

La estructura \textit{if-else} puede ser simple, es decir, que solo incluya \textbf{if} y no \textbf{else}. Recordemos que Python no usa puntos y comas (;) ni llaves (\{\}) para encerrar ciertos bloques bajo ciertas funciones, procedimientos, condiciones, ciclos o clases, usa la \textbf{indentación}, y las expresiones a evaluar de \textit{if} pueden contener variables tipo booleano, operaciones con operaciones aritméticas o comparación con operadores relacionales. No olvide los \textbf{dos puntos} (:) al final de la instrucción if y else, y que las sentencias ifs pueden ser anidadas (un if dentro de otro), como vemos a continuación:
\begin{lstlisting}
	x = 20 #Declara e inicializa variable.
	
	if (x == 0): #Primer condicion.
		print("Cero") #Despliega mensaje.
	else: #Si no se cumplio la condicion...
		if (x > 0): #Condicion anidada.
			print("Mayor a cero")
		else:
			if (x >= 10): #Condicion anidada.
				print("Mayor a 10")
			else:
				print("Numero no valido")
\end{lstlisting}

Vemos que anidar de esa manera ifs se ve algo mal, por cierte, existe la estructura \textbf{elif}, la cual es una abreviación de \textbf{else if}, estructura conocido en otros lenguajes de programación, repetimos el ejemplo pero con elifs:
\begin{lstlisting}
	x = 20 #Declara e inicializa variable.
	
	if (x == 0): #Primer condicion.
		print("Cero") #Despliega mensaje.
	elif (x > 0): #Segunda condicion con elif.
		print("Mayor a cero")
	elif (x >= 10): #Tercera condicion con elif.
		print("Mayor a 10")
	else: #Si no se cumplio ninguna condicion.
		print("Numero no valido")
\end{lstlisting}

No olvide que puede utilizar \textbf{operadores lógicos} para juntar más de una condición en una instrucción a evaluar:
\begin{lstlisting}
	x = 20 #Declara e inicializa variable.
	
	if (x > 0 and x <= 10): #Condicion con operadores relacionales y logicos.
		print("Numero valido") #Imprime mensaje.
	else: #Si no se cumple la condicion...
		print("Numero no valido") #Imprime mensaje.
\end{lstlisting}

La condición del ejemplo anterior puede ser simplificado de la siguiente manera:
\begin{lstlisting}
	x = 20 #Declara e inicializa variable.
	
	if (0 < x < 10): #Condicion simplificada.
		print("Numero valido") #Imprime mensaje.
	else: #Si no se cumple la condicion...
		print("Numero no valido") #Imprime mensaje.
\end{lstlisting}

Si leemos la condición, vemos que 0 debe ser menor a x, y x debe ser menor a 10, así lo interpreta el lenguaje y la evalua, con este pensamiento, podemos simplificar otras condiciones numéricas.



\section{Ciclos}
\hspace{0.55cm}Al igual que las estructuras condicionales, los \textbf{ciclos} deben tener \textbf{dos puntos} (:) al final de su declaración e \textbf{indentación} para su correcto funcionamiento.


\subsection{Ciclo While}
\hspace{0.55cm}Ciclo que se repite mientras una condición se cumpla, debe ser cuidadoso, ya que debe tener algún tipo de variable que haga que el ciclo pare, sino, el ciclo continuará indefinidamente, causando problemas tanto en el programa, como en la máquina donde se está ejecutando. Estructura:
\begin{lstlisting}
	while (condicion):
		#Instrucciones.
	
	
	x = 0 #Declara e inicializa variable.
	
	while (x < 9): #Ciclo que se repite 10 veces.
		print(x) #Imprime x.
		x -= 1 #Variable que hace que el ciclo termine.
		
	print("Terminado") #Mensaje de fin de ciclo.
\end{lstlisting}

Al igual que las estructuras if, la condición del ciclo while puede contener distintas condiciones que conforman una expresión a evaluar, puede usar operadores relacionales y lógicos y contener estructuras if dentro del ciclo.


\subsubsection{break y continue}
\hspace{0.55cm}Estas son palabras reservadas para manipular el flujo de ciclos while, en caso de que queramos terminar abruptamente un ciclo while, podemos usar la instrucción \textbf{break}, por ejemplo, si queremos que el usuario ingrese diez números, pero si uno de ellos es menor a cero, se termina el ciclo; la instrucción \textbf{continue} puede decirse que "salta" la iteración actual del ciclo y pasa al siguiente, no para como tal al ciclo, si se ejecuta \textit{continue}, toma en cuenta las siguientes instrucciones a continue y pasa a la siguiente iteración. Ejemplos de estas dos instrucciones:
\begin{lstlisting}
	x = 0 #Declara e inicializa variable.
	
	while (x < 9): #Ciclo que se repite 10 veces.
		num = int(input("Ingrese un numero entre 0 y 9: ")) #Entrada de un numero.
		
		if (num < 0): #Si el numero es menor a cero, termina el ciclo con break.
			print("Numero no valido.")
			break
		
		if (num > 9) #Si el numero es mayor a 9, no se toma en cuenta el numero y pasa a la siguiente iteracion.
			print("Numero muy elevado")
			x += 1
			continue

		print("El numero que ingresaste es: " + num)
		x += 1
	print("Terminado") #Mensaje de fin de ciclo.
\end{lstlisting}


\subsection{Ciclo For}
\hspace{0.55cm}Ciclo que se repite una cantidad fija de veces, se utiliza para recorrer listas o cadena de texto, se pueden utilizar las instrucciones \textbf{break} y \textbf{continue} para salir o saltar en el ciclo, como con el ciclo while. Sintáxis:
\begin{lstlisting}
	#Declara lista y cadena de ejemplo.
	lista = [1, 2, 3, 4]
	nombre = "esto es una prueba con ciclo for"
	
	for x in lista: #Declara una variable que represente los elementos de la lista.
		print(x) #Imprime los elementos de la lista.
		
	for y in nombre #Declara una variable que represente los caracteres de la cadena.
		print(y) #Imprime caracter a caracter de la cadena de texto.
\end{lstlisting}


\subsection{Ciclo Do-While}
\hspace{0.55cm}Como tal, no existe este tipo de ciclo en Python, sin embargo, podemos crearlo o simularlo. Para lograr esto, utilizamos un ciclo While, donde la condición que evalúe simplemente sea un valor booleano True, con esto, el ciclo entra automáticamente y cumple con la característica de que el ciclo al menos da una vuelta. Es necesario que haya una condición dentro del ciclo que le de fin al mismo, para evitar un ciclo indeterminado, veamos el siguiente ejemplo:
\begin{lstlisting}
	# Variable contador.
	x = 1
	
	# Condición cierta, entra al ciclo.
	while True:
		# Imprime el valor de la variable.
		print(x)
		# Incrementa el valor de la variable.
		x = x + 1
		# Si el valor de la variable es 10, rompe el ciclo.
		if x == 10
			break
\end{lstlisting}

Al tener un valor booleando True, la condición es cierta y entra al ciclo, si la variable contador llega a cierto valor o condición, se rompe el ciclo con la sentencia \textbf{break}, con esto, aseguramos que se de una vuelta al menos dentro del ciclo y aseguramos que el ciclo tenga un fin.



\section{Sentencia Switch}
\hspace{0.55cm}En Python no existe esta estructura condicional, la forma de simularlo es un conjunto de \textit{if-elif's}, también se pueden utilizar diccionarios, donde su llave es el índice que se evalúa y su valor es una función que contiene el valor que queremos mostrar, pero para no complicarlos la vida, nos quedaremos con la serie de if-elif's.