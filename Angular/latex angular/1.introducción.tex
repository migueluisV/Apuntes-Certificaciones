\section{Introducción}


\subsection{¿Qué es Angular?}

Angular es utilizado para crear sitios web escalables con contenido dinámico que cambia con base en las interacciones del usuario.

Angular es un Framework y una plataforma de desarrollo, esto quiere decir que provee de un alto rango de herramientas que te ayudan a escribir, visualizar y desplegar un proyecto, al mismo tiempo que es una estructura sobre la cual puedes construir un proyecto y personalizarlo.

Angular se basa principalmente en HTML, CSS y Typescript.


\subsection{Instalación y creación de proyectos}


\subsubsection{Instalación}

Para instalar Angular en una máquina local, es necesario insertar el siguiente comando:
\begin{lstlisting}[style=bash]
npm install -g @angular/cli
\end{lstlisting}


\subsubsection{Creación de un proyecto}

Para crear un nuevo proyecto de Angular, se escribe el siguiente comando, donde nombreProyecto es el nombre que le daremos al proyecto:
\begin{lstlisting}[style=bash]
ng new nombreProyecto
\end{lstlisting}


\subsubsection{Ejecución del proyecto}

Para ejecutar el proyecto, se debe de dirigir a la carpeta donde este se haya creado y ejecutar el siguiente comando:
\begin{lstlisting}[style=bash]
ng serve -o
\end{lstlisting}


\subsection{Estructura de archivos del proyecto}

La estructura de archivos de los proyectos Angular es la siguiente (depende también del IDE o  editor de texto que se utilice, la estructura siguiente fue tomada de haber creado un proyecto con Visual Studio Code):
\begin{itemize}
    \item src
    \begin{itemize}
        \item app
        \begin{itemize}
            \item app.component.css
            \item app.component.html
            \item app.component.spec.ts
            \item app.component.ts
            \item app.config.server.ts
            \item app.config.ts
            \item app.routes.ts
        \end{itemize}
        \item assets
        \begin{itemize}
            \item .gitkeep
        \end{itemize}
        \item favicon.ico
        \item index.html
        \item main.server.ts
        \item main.ts
        \item styles.css
    \end{itemize}
    \item .editorconfig
    \item .gitignore
    \item angular.json
    \item package.json
    \item package-lock.json
    \item README.md
    \item server.ts
    \item tsconfig.app.json
    \item tsconfig.json
    \item tsconfig.scep.json
\end{itemize}

Donde:
\begin{itemize}
    \item index.html es la estructura HTML principal del proyecto.
    \item main.ts es el punto de entrada del proyecto cuando este es ejecutado.
    \item styles.css es la hoja de estilos CSS principal del proyecto.
    \item app.component.ts y app.component.html representan el componente raíz del proyecto.
\end{itemize}
