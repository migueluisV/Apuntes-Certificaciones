\section{Filtrado de datos}

Anteriormente, seleccionábamos \textbf{todos} los datos de toda una tabla o columna. El comando \textbf{WHERE} permite seleccionar únicamente uno o varios registros que cumplan con una condición. Los operadores pueden ser consultados en la \textit{Tabla \ref{tab: 9}}:
\begin{table}[H]
    \centering
    \caption{Operadores relacionales en SQL}
    \label{tab: 9}
    \begin{tabular}{l l}
        \hline
        \textbf{Operador} & \textbf{Descripción} \\
        \hline
        =           & Igual a \\
        $>$         & Mayor a \\
        $<$         & Menor a \\
        $>$=        & Mayor o igual a \\
        $<$=        & Menor o igual a \\
        $<>$        & Distinto a \\
        !=          & Distinto a  (en algunas versiones de SQL) \\
        \hline
    \end{tabular}
\end{table}

Se pueden utilizar cadenas dentro del filtrado, se especifica el valor de la misma con comillas simples (''), en caso de que el registro contenga una comilla simple (por ejemplo: 'let's'), agregue otra comilla ('let''s'). Las siguientes sentencias son ejemplos de filtrado de datos con operadores lógicos, copie y pegue en su SGBD para poder apreciar los resultados:
\begin{lstlisting}
    SELECT * FROM Personas WHERE edad = 60;
    SELECT * FROM Personas WHERE edad $<>$ 60;
    SELECT * FROM Personas WHERE nombre = 'David';
    SELECT * FROM Personas WHERE edad < 30
\end{lstlisting}

Los comandos \textbf{BETWEEN} y \textbf{AND} sirven para seleccionar un rango de registros resultantes de una sentencia con el comando WHERE, un ejemplo se ve en la \textit{Tabla \ref{tab: 10}}:
\begin{lstlisting}
    SELECT * FROM Personas
    WHERE edad BETWEEN 30 AND 60
\end{lstlisting}
\begin{table}[H]
    \centering
    \caption{Seleccionando un rango de registros con WHERE, BETWEEN y AND}
    \label{tab: 10}
    \begin{tabular}{|l|l|l|l|l|}
        \hline
        \textbf{id} & \textbf{nombre} & \textbf{apellidos} & \textbf{ciudad} & \textbf{edad} \\
        \hline
        2 & David       & Williams  & Los Angeles   & 42 \\
        \hline
        3 & Chloe       & Anderson  & Chicago       & 65 \\
        \hline
        5 & James       & Roberts   & Philadelphia  & 31 \\
        \hline
        7 & Daniel      & Harris    & Los Angeles   & 67 \\
        \hline
        8 & Charlotte   & Walker    & Chicago       & 45 \\
        \hline
    \end{tabular}
\end{table}

\textit{Nota}: los valores límite (30 y 60 para el ejemplo anterior) son incluidos en los registros resultantes.



\section{Condiciones lógicas}

Puede combinar operadores relacionales con operadores lógicos, estos pueden ser consultados en la \textit{Tabla \ref{tab: 11}}:
\begin{table}[H]
    \centering
    \caption{Operadores lógicos en SQL}
    \label{tab: 11}
    \begin{tabular}{m{3cm} m{10cm}}
        \hline
        \textbf{Operador} & \textbf{Descripción} \\
        \hline
        ALL     & Regresa TRUE si todos los valores de una sub-sentencia coinciden con la condición \\
        AND     & Regresa TRUE si las dos condiciones separadas por el operador AND son ciertas \\
        ANY     & Regresa TRUE si alguno de los valores de una sub-sentencia coincide con la condición \\
        BETWEEN & Regresa TRUE si el operador está dentro de un rango de comparación \\
        EXIST   & Regresa TRUE si la sub-sentencia regresa uno o varios registros \\
        IN      & Regresa TRUE si el operador es igual a un item dentro de una lista de expresiones \\
        LIKE    & Regresa TRUE si el operador coincide con el patrón \\
        NOT     & Muestra el contrario o negación de una condición verdadera \\
        OR      & Regresa TRUE si alguna de las dos condiciones separadas por el operador OR son ciertas \\
        SOME    & Regresa TRUE si alguno de los valores de la sub-sentencia coincide con la condición \\
        \hline
    \end{tabular}
\end{table}

Ya se utilizaron los operadores \textbf{BETWEEN} y \textbf{AND} en el tema anterior para conseguir el rango de valores aceptados en la condición. El operador AND permite juntar dos condiciones, si ambas son ciertas, se regresa \textbf{TRUE}. El operador \textbf{OR} sigue la misma lógica, solamente que si alguna de ambas condiciones es cierta, se regresa TRUE. A continuación, ejemplos:
\begin{lstlisting}
    SELECT * FROM Personas
    WHERE ciudad = 'Los Angeles' AND edad = 42;
    SELECT * FROM Personas
    WHERE ciudad = 'Los Angeles' OR ciudad = 'Chicago'
\end{lstlisting}

El operador \textbf{IN} funciona para evitar utilizar varias veces el operador \textbf{OR}, como se ve en el siguiente ejemplo:
\begin{lstlisting}
    SELECT * FROM Personas
    WHERE ciudad
    IN ('Los Angeles', 'Chicago')
\end{lstlisting}

Esta consulta regresa todos los registros que tengan el atributo "ciudad" con los valores "Los Angeles" y "Chicago". Es obligatorio que se utilicen los paréntesis, las comillas simples y separación por comas para todos los valores a utilizar.

Caso contrario, el operador \textbf{NOT} regresará todos los registros con el atributo "ciudad" distinto a "Los Angeles" y "Chicago", es decir, todos los registros contrarios a \textbf{TRUE} o los registros contrarios a los escritos.



\section{Filtros de texto}

Vimos anteriormente que se pueden seleccionar registros en base a la cadena de un atributo, a esta característica se podemos sumar que se pueden recoger registros en base a un patrón. Para lograr este cometido, utilizamos el operador \textbf{LIKE}, en conjunto con los caracteres comodines (Wildcard Characters) y las comillas simples para encerrar el patrón, uno de los más populares es el comodín \%, el cual representa uno o varios caracteres que, para este caso, son ignorados y solamente se toma en cuenta la cadena o caracteres a buscar como patrón, puede consultar más información sobre los comodines en este \href{https://www.w3schools.com/sql/sql_wildcards.asp}{enlace}. Veamos un ejemplo para que se entienda un poco más como se usan estos comodines:
\begin{center}
    \textit{
        '\%sos' ignora los primeros caracteres menos 'sos'. \\
        'be\%' ignora los caracteres siguientes a 'be'. \\
        '\% san \%' ignora los caracteres antes y después de 'san'.
    }
\end{center}

Entonces, con este comodín podemos ignorar \textit{n} cantidad de caracteres y tomar solamente los que nos interesan, siendo esto un patrón que utiliza el operador \textbf{LIKE} para recoger registros con una secuencia de caracteres específicos en una cadena. Veamos varios ejemplos con la tabla de ejemplo que utilizamos y los resultados aparecen en las siguientes tablas:
\begin{lstlisting}
    SELECT * FROM Personas
    WHERE apellidos
    LIKE '\%s'
\end{lstlisting}
\begin{table}[H]
    \centering
    \caption{Seleccionando registros con una "s" al final}
    \label{tab: 12}
    \begin{tabular}{|l|l|l|l|l|}
        \hline
        \textbf{id} & \textbf{nombre} & \textbf{apellidos} & \textbf{ciudad} & \textbf{edad} \\
        \hline
        2 & David       & Williams  & Los Angeles   & 42 \\
        \hline
        4 & Emily       & Adams     & Houston       & 29 \\
        \hline
        5 & James       & Roberts   & Philadelphia  & 31 \\
        \hline
        6 & Andrew      & Thomas    & New York      & 21 \\
        \hline
        7 & Daniel      & Harris    & Los Angeles   & 67 \\
        \hline
    \end{tabular}
\end{table}
\begin{lstlisting}
    SELECT * FROM Personas
    WHERE apellidos
    LIKE '\%a\%'
\end{lstlisting}
\begin{table}[H]
    \centering
    \caption{Seleccionando registros con una "a" en medio}
    \label{tab: 13}
    \begin{tabular}{|l|l|l|l|l|}
        \hline
        \textbf{id} & \textbf{nombre} & \textbf{apellidos} & \textbf{ciudad} & \textbf{edad} \\
        \hline
        2 & David       & Williams  & Los Angeles   & 42 \\
        \hline
        4 & Emily       & Adams     & Houston       & 29 \\
        \hline
        6 & Andrew      & Thomas    & New York      & 21 \\
        \hline
        7 & Daniel      & Harris    & Los Angeles   & 67 \\
        \hline
        8 & Charlotte   & Walker    & Chicago       & 45 \\
        \hline
    \end{tabular}
\end{table}

El primer ejemplo selecciona todos los registros donde el apellido de una persona tenga una "s" al final, mientras que el segundo ejemplo selecciona todos los registros donde el apellido de la persona tenga una "a" en medio.

Este comodín es muy poderoso, podemos usar el comodín \_ para seleccionar únicamente un carácter dentro del patrón, como se ve en la \textit{Tabla \ref{tab: 14}}:
\begin{lstlisting}
    SELECT * FROM Personas
    WHERE apellidos
    LIKE '\_ork'
\end{lstlisting}
\begin{table}[H]
    \centering
    \caption{Seleccionando registros con "ork" al final}
    \label{tab: 14}
    \begin{tabular}{|l|l|l|l|l|}
        \hline
        \textbf{id} & \textbf{nombre} & \textbf{apellidos} & \textbf{ciudad} & \textbf{edad} \\
        \hline
        1 & John        & Smith     & New York      & 24 \\
        \hline
        6 & Andrew      & Thomas    & New York      & 21 \\
        \hline
    \end{tabular}
\end{table}



\section{NULL}

La palabra reservada \textbf{NULL} representa la ausencia de un valor en una tabla. Cuando hacemos un registro dejamos vacío un atributo de la tabla, SQL internamente le asigna NULL a ese valor vacío, NULL es diferente de un espacio en blanco y cero. Podemos comprobar si un registro tiene o no un atributo NULL con la combinación de comandos:
\begin{lstlisting}
    SELECT * FROM Personas
    WHERE apellido IS NULL;
    SELECT * FROM Personas
    WHERE apellido IS NOT NULL;
\end{lstlisting}
