\section{Conceptos básicos}

\textbf{JavaScript} es uno de los lenguajes de programación más populares, utilizado popularmente para crear sitios web dinámicos e interactivos, pero también es usado para crear aplicaciones de celular, videojuegos, procesamiento de datos y más.


\subsection{Salidas}

\textbf{Dentro de un documento HTML}

Una cosa es desplegar un mensaje en un archivo \textit{.js} y otra en un archivo \textit{.html}, con la función \textbf{document.write()} escribimos una cadena dentro de la etiqueta \textbf{<script>} en un archivo HTML.
\begin{lstlisting}
    <script>
        document.write("Hola mundo.")
    </script>
\end{lstlisting}

Gracias a que estamos escribiendo por medio del lenguaje de etiquetas HTML, podemos aplicar etiquetas del mismo al acabado de nuestro mensaje:
\begin{lstlisting}
    <script>
        <!- El mensaje estará escrito en negrita y cursiva. ->
        document.write("<br><i>Hola mundo.</i></br>")
    </script>
\end{lstlisting}

\textit{Nota}: es recomendable utilizar esta función únicamente para salidas de pruebas u errores.

\textbf{En la consola del buscador}

Para escribir un mensaje en la consola del navegador, utiliza el comando \textbf{console.log()}, el \textbf{texto} que vaya a ser escrito debe estar encerrado \textbf{entre comillas sencillas o dobles} ('texto', "texto").
\begin{lstlisting}
    \console.log("Esto es un mensaje.")
\end{lstlisting}

Este tipo de salida es más utilizada para pruebas y probar el funcionamiento del código, se recomienda esta función contra la mencionada anteriormente.


\subsection{Variables}

Para declarar una variable se utiliza la palabra reservada \textbf{var}:
\begin{lstlisting}
    var x = 10;
\end{lstlisting}

\textit{Nota}: JavaScript diferencia variables con el mismo nombre pero distinta cantidad de variables, "nombre" y "Nombre" son dos variables distintas.

Algunas reglas para la declaración de variables en este lenguaje son:
\begin{enumerate}
    \item El primer carácter de una variable debe ser una letra, \textbf{guión bajo} (\_) o un \textbf{símbolo de dolar} (\$).
    \item El primer carácter de una variable no puede ser un número.
    \item Los nombres de variables no pueden incluir operadores matemáticos o lógicos.
    \item Los nombres de variables no pueden contener espacios en blanco.
    \item Los nombres de variables no pueden contener símbolos especiales (", \#, \%, \&, etc).
\end{enumerate}


\subsection{Comentarios}

Para comentar una sola línea de código se utilizan \textbf{dos diagonales} (\textbf{//}) y para comentar múltiples líneas se utiliza los caracteres \textbf{/*} al inicio de las instrucciones que buscas comentar, y \textbf{*/} al final de las instrucciones.
\begin{lstlisting}
    // Esto es un comentario.
    alert("Mensaje dentro de una alerta.")
    /*
    Esto
    También
    Es
    Un
    Comentario.
    */
\end{lstlisting}


\subsection{Tipos de datos}

En este lenguaje, no es necesario declarar una variable con su tipo de dato, sin embargo, no es una buena práctica declarar una variable con un entero, y algunas instrucciones después, asignarle una cadena de caracteres.
\begin{lstlisting}
    // Declaración de variables.
    var x = 1;
    var y = 1.1;
    var z = 1.1111;
    x = "Esto es una variable"; // Esto no es correcto.
\end{lstlisting}

Podemos utilizar una sola comilla (') o dobles comillas (") para contener un texto dentro de una variable, a su vez, podemos utilizar los escapes \textbackslash " y \textbackslash ' para utilizar dichas comillas dentro de una cadena.
\begin{lstlisting}
    var nombre = "mi nombre es \"mario\"";
    var apellido = 'mi nombre es \'casas\''
    var edad = "mi edad es '21'";
\end{lstlisting}

\textit{Nota}: no es necesario utilizar caracteres de escape de una comilla dentro de dobles comillas, ni dobles comillas dentro de comillas.

Algunos caracteres de escape que podemos utilizan se ven en la \textit{Tabla \ref{tab: 1}}:
\begin{table}[H]
    \begin{center}
        \caption{Caracteres de escape válidos}
        \label{tab: 1}
        \begin{tabular}{c l}
            \hline
            \textbf{Carácter de escape}&\textbf{Función} \\
            \hline
            \textbackslash ' & Una comilla \\
            \textbackslash " & Doble comilla \\
            \textbackslash \textbackslash & Diagonal \\
            \textbackslash n & Salto de línea \\
            \textbackslash r & Posiciona el cursor al inicio de la línea \\
            \textbackslash t & Tabulación \\
            \textbackslash b & Posiciona el cursos un carácter atrás en el texto o consola \\
            \textbackslash f & Genera un salto de página \\
            \hline
        \end{tabular}
    \end{center}
\end{table}

Los \textbf{valores booleanos} son: \textbf{true} y \textbf{false}, el primero para casos positivos o reales, el segundo para valores como 0, null, indefinido o cadenas vacías.


\subsection{Operadores}


\subsubsection{Operadores aritméticos}

La \textit{Tabla \ref{tab: 2}} contiene los operadores aritméticos válidos en este lenguaje:
\begin{table}[H]
    \begin{center}
        \caption{Operadores aritméticos en JavaScript}
        \label{tab: 2}
        \begin{tabular}{c l}
            \hline
            \textbf{Operador}&\textbf{Definición} \\
            \hline
            + & Suma o Concatenación \\
            - & Resta \\
            $\ast$ & Multiplicación \\
            / & División \\
            \% & Modulo (residuo de una división) \\
            ++ & Incremento \\
            -- & Decremento \\
            \hline
        \end{tabular}
    \end{center}
\end{table}

La función \textbf{eval()} toma una cadena que contiene una expresión aritmética y regresa su resultado:
\begin{center}
    \textit{console.log(eval("2 + 2")); // Imprime 4.}
\end{center}

Al igual que en otros lenguajes, JavaScript posee los operadores de incremento y decremento post y pre:
\begin{center}
    \textit{
            var++ (incrementa después de una instrucción) \\
            ++var (incrementa antes de una instrucción) \\
            var-- (decrementa después de una instrucción) \\
            --var (decrementa antes de una instrucción) \\
    }
\end{center}


\subsubsection{Operadores de asignación}

La \textit{Tabla \ref{tab: 3}} contiene los operadores de asignación válidos en este lenguaje:
\begin{table}[H]
    \begin{center}
        \caption{Operadores de asignación en JavaScript}
        \label{tab: 3}
        \begin{tabular}{c l}
            \hline
            \textbf{Operador}&\textbf{Equivalencia} \\
            \hline
            = & x = y \\
            += & x = x + y \\
            -= & x = x - y \\
            *= & x = x * y \\
            /= & x = x / y \\
            \%= & x = x \% y \\
            \hline
        \end{tabular}
    \end{center}
\end{table}

\textit{Nota}: pueden combinarse el uso de varios operadores de asignación en una sola instrucción.


\subsubsection{Operadores de comparación}

La \textit{Tabla \ref{tab: 4}} contiene los operadores de comparación válidos en este lenguaje:
\begin{table}[H]
    \begin{center}
        \caption{Operadores de comparación en JavaScript}
        \label{tab: 4}
        \begin{tabular}{c l}
            \hline
            \textbf{Operador}&\textbf{Definición} \\
            \hline
            == & Igual a \\
            === & Idénticos (iguales o del mismo tipo) \\
            != & No igual a \\
            !== & No idéntico \\
            $>$ & Mayor \\
            $>$= & Mayor igual \\
            $<$ & Menor \\
            $<$= & Menor igual \\
            \hline
        \end{tabular}
    \end{center}
\end{table}

Las comparaciones regresan true o false si son ciertas o no.


\subsubsection{Operadores lógicos y booleanos}

La \textit{Tabla \ref{tab: 5}} contiene los operadores lógicos válidos en este lenguaje:
\begin{table}[H]
    \begin{center}
        \caption{Operadores de comparación en JavaScript}
        \label{tab: 5}
        \begin{tabular}{m{3cm} m{10cm}}
            \hline
            \textbf{Operador}&\textbf{Definición} \\
            \hline
            \&\& & Y. Regresa true si ambas expresiones son verdaderas \\
            $||$ & O. Regresa true si una de las expresiones es verdadera \\
            ! & Negación. Regresa el valor contrario (true o false) al resultado de la expresión \\
            \hline
        \end{tabular}
    \end{center}
\end{table}

Este lenguaje soporta el uso del \textbf{operador ternario}:
\begin{lstlisting}
    var mayorDeEdad = (edad < 18) ? "Muy joven" : "Muy viejo";
\end{lstlisting}

Como vimos, está constituido de una condición, el símbolo de pregunta, su primer valor, dos puntos y el segundo valor; solamente soporta dos valores (true y false), a diferencia de una sentencia if, que puede tener varios \textit{if´s anidados} o \textit{else if}.
\begin{lstlisting}
    variable = (condición) ? valor1 : valor2
\end{lstlisting}
