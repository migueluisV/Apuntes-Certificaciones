\section{Evaluación de expresiones}

\subsection{Sentencia if else}
\hspace{0.55cm}Evalúa una expresión, si esta es acertada regresa true, si no lo es, regresa false. Su estructura es la siguiente:
\begin{lstlisting}
    if (condición) {
        // Instrucciones si la condición es verdadera.
    }
    else {
        // Instrucciones si la condición es falsa.
    }
\end{lstlisting}

\textit{Nota}: el lenguaje soporta que, si solamente se escribe una instrucción en cualquiera de sus bloques, no sea necesario escribir las llaves y el bloque \textit{else} puede ser omitido.


\subsection{Sentencia else if}
\hspace{0.55cm}En ocasiones, es requerido que se evalúe una condición si una condición previa fue falsa, para eso funciona la sentencia else if:
\begin{lstlisting}
    if (condición 1) {
        // Instrucciones si la condición es verdadera.
    }
    else if (condición 2) {
        // Instrucciones si la condición 1 es falsa.
    } else {
        // Instrucciones si la condición 2 es falsa.
    }
\end{lstlisting}


\subsection{Sentencia switch}
\hspace{0.55cm}Evalúa una variable y se le asigna un valor dependiendo de otra cantidad de valores o parámetros. Su estructura es la siguiente:
\begin{lstlisting}
    switch (variable o expresión) {
        case n1:
            // Instrucciones.
            break;
        case n2:
            // Instrucciones.
            break;
        .
        .
        .
        case n:
            // Instrucciones.
            break;
        default:
            // Instrucciones.
    }
\end{lstlisting}

\textit{Nota}: el valor \textbf{default} puede ser omitido y no es obligatoria la palabra reservada \textbf{break}.



\section{Ciclos}

\subsection{Ciclo For}
\hspace{0.55cm}Ejecuta un bloque de código n cantidad de veces. Su estructura es:
\begin{lstlisting}
    for (inicializador o contador; condición para ejecución; incremento o decremento) {
        // Instrucciones.
    }
\end{lstlisting}

Donde:
\begin{itemize}
    \item \textbf{inicializador o contador}: es la variable que será incrementada o decrementada a lo largo de la ejecución del bloque de código. Puede ser omitido siempre y cuando haya alguna variable fuera de la declaración del ciclo que maneje la ejecución del mismo, a su vez, pueden haber múltiples contadores dentro de la declaración.
    \item \textbf{condición para ejecución}: condición que permite que el bloque se ejecute; si la condición ya no es cierta, se sale del ciclo.
    \item \textbf{incremento o decremento}: incrementa o decrementa la variable contador cuando una vuelta se da en el ciclo.
\end{itemize}

Vemos a continuación dos ejemplos del primer punto anterior mencionados:
\begin{lstlisting}
    i = 1;
    // Ciclo for sin un contador.
    for (; i < 5; i++) {}
    // Ciclo for con dos contadores o inicializadores.
    for (x = 1, text = ""; x < 5; x++) {}
\end{lstlisting}


\subsection{Ciclo While}
\hspace{0.55cm}Ejecuta un bloque de código mientras una condición sea verdadera. Su estructura es:
\begin{lstlisting}
    while (condición) {
        // Instrucciones.
    }
\end{lstlisting}

Debe poseer una variable contador dentro de su bloque para manejar las vueltas dentro del ciclo y su fin es que el ciclo termine, sino, sería un ciclo infinito.


\subsection{Ciclo Do-While}
\hspace{0.55cm}Ejecuta un bloque de código mientras una condición sea verdadera y se ejecuta mínimo una vez. Su estructura es:
\begin{lstlisting}
    do {
        // Instrucciones.
    }
    while (condición);
\end{lstlisting}


\subsection{Break y Continue}
\hspace{0.55cm}La palabra reservada \textbf{break} es utilizada para terminar la ejecución de un ciclo, aún si su condición todavía era verdadera como para continuar con las vueltas.\\
La palabra reservada \textbf{continue} es utilizada para saltar n cantidad de líneas de código después de la palabra reservada en cuestión, es decir, salta una vuelta del ciclo. Veremos un ejemplo de ambas a continuación:
\begin{lstlisting}
    for (x = 1; x <= 10; x++) {
        if (x == 5){
            continue;
        }
        console.log(x);
        if (x == 8){
            break;
        }
    }

    /*
    Imprime:
    1
    2
    3
    4
    6
    7
    8
    */
\end{lstlisting}

En el código anterior se imprimen los primeros cuatro números, se salta el número cinco por la sentencia \textit{continue} y continua imprimiendo, cuando x vale ocho, termina el ciclo.