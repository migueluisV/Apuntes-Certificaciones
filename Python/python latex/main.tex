% Tipo de documento y paquetes a utilizar.
\documentclass[12pt]{article}
\usepackage[utf8]{inputenc}
%\usepackage{amsmath, amsthm, amsfonts, mathtools}	% Paquete para usar más fórmulas y ecuaciones.
\usepackage{graphicx}								% Paquete para usar imágenes y figuras.
\usepackage{geometry}								% Paquete para trabajar con los márgenes del documento.
\usepackage{fancyhdr}								% Paquete para personalizar encabezado y pie de página.
\usepackage{lastpage}								% Paquete para reverenciar páginas del documento.
\usepackage{listings}								% Paquete para escribir código de programación.
\usepackage{inconsolata}							% Paquete de tipo de letra Consola.
\usepackage{multirow}								% Paquete para combinar filas y columnas en tablas.
\usepackage{array}									% Paquete para trabajar tablas especializadas.
\usepackage{xcolor}									% Paquete básico para agregar color al texto.
\usepackage{float}          						% paquete para utilizar fijación de figuras H.
\usepackage{hyperref}                               % Paquete para insertar links en el documento.

% Define colores nuevos
\definecolor{color}{HTML}{E4E4EE}
\definecolor{verde}{HTML}{3C8031}

% Personalización de la fuente para el código.
\lstset{
    language = Python,                      % Lenguaje con palabras reservadas de este resaltadas.
    basicstyle = \ttfamily\footnotesize,    % Utiliza la fuente tttfamily, en especial el paquete inconsolata.
    frame = single,					        % Quita el marco al cuadro flotante que contiene el código o texto.
    backgroundcolor = \color{color},        % Cambia el color del fondo del marco del código. Utiliza el paquete "xcolor" y define un nuevo color.
    columns = fullflexible,                 % Ajusta el cuadro flotante al tamaño del texto del documento.
    breaklines = true,                      % Ajusta el texto dentro del contenedor.
    inputencoding = utf8,                   % Admite caracteres del código UTF8.
    extendedchars = true,                   % Soporte para caracteres especiales.
    %numbers = left,                        % Agrega número de línea al código (izquierda, sin número y derecha).
    showstringspaces = false,               % Quita los guiones bajos predeterminados de los espacios en cadenas de texto.
    escapebegin = \obeyspaces,              % Complemento de la entrada anterior.
    %numberstyle = \color{red},             % Color de los números en el texto o código.
    %stringstyle = \color{red},             % Color de las cadenas de texto en el texto o código.
    %keywordstyle = \color{red},            % Color de las palabras reservadas en el texto o código.
    %identifierstyle = \color{red},         % Color del texto o código.
    commentstyle = \color{verde},           % Color de los comentarios en el texto o código.
    literate =                              % Acepta los siguientes caracteres especiales fuera de UTF8.
            {á}{{\'a}}1 {é}{{\'e}}1 {í}{{\'i}}1 {ó}{{\'o}}1 {ú}{{\'u}}1
            {Á}{{\'A}}1 {É}{{\'E}}1 {Í}{{\'I}}1 {Ó}{{\'O}}1 {Ú}{{\'U}}1
            {ñ}{{\~n}}1 {Ñ}{{\~N}}1,
}

% Márgenes del documento.
\newgeometry{
    top=2.5cm,      %superior
    bottom=2.5cm,   %inferior
    outer=2.5cm,    %parte exterior
    inner=2.5cm,    %parte interior
}

% Personalización de la cabecera y pie de página.
\pagestyle{fancy}
\fancyhf{}
% Texto en esquina superior derecha.
\rhead{Texmaker}
% Texto en esquina superior izquierda.
\lhead{Apuntes de Python}
% Texto en esquina inferior derecha (Página n de n).
\rfoot{Pagina \thepage \hspace{1pt} de \pageref{LastPage}}
% Ancho de línea horizontal superior e inferior.
\renewcommand{\headrulewidth}{1pt}
\renewcommand{\footrulewidth}{1pt}

% Datos de la portada.
\title{Apuntes de Python}
\author{migueluisV}
\date{Actualizadas: Octubre 2022}

% Inicio del documento.
\begin{document}

% cambia los títulos de los índices:
% Content - Índice
% List of Tables - Índice de Tablas
\renewcommand*\contentsname{Índice}
\renewcommand{\listtablename}{Índice de Tablas}

% Agrega portada e índices.
\maketitle\newpage
\tableofcontents\newpage
\listoftables\newpage

% Archivos que conforman al documento.
\section{Conceptos básicos}

\textbf{JavaScript} es uno de los lenguajes de programación más populares, utilizado popularmente para crear sitios web dinámicos e interactivos, pero también es usado para crear aplicaciones de celular, videojuegos, procesamiento de datos y más.


\subsection{Salidas}

\textbf{Dentro de un documento HTML}

Una cosa es desplegar un mensaje en un archivo \textit{.js} y otra en un archivo \textit{.html}, con la función \textbf{document.write()} escribimos una cadena dentro de la etiqueta \textbf{<script>} en un archivo HTML.
\begin{lstlisting}
    <script>
        document.write("Hola mundo.")
    </script>
\end{lstlisting}

Gracias a que estamos escribiendo por medio del lenguaje de etiquetas HTML, podemos aplicar etiquetas del mismo al acabado de nuestro mensaje:
\begin{lstlisting}
    <script>
        <!- El mensaje estará escrito en negrita y cursiva. ->
        document.write("<br><i>Hola mundo.</i></br>")
    </script>
\end{lstlisting}

\textit{Nota}: es recomendable utilizar esta función únicamente para salidas de pruebas u errores.

\textbf{En la consola del buscador}

Para escribir un mensaje en la consola del navegador, utiliza el comando \textbf{console.log()}, el \textbf{texto} que vaya a ser escrito debe estar encerrado \textbf{entre comillas sencillas o dobles} ('texto', "texto").
\begin{lstlisting}
    \console.log("Esto es un mensaje.")
\end{lstlisting}

Este tipo de salida es más utilizada para pruebas y probar el funcionamiento del código, se recomienda esta función contra la mencionada anteriormente.


\subsection{Variables}

Para declarar una variable se utiliza la palabra reservada \textbf{var}:
\begin{lstlisting}
    var x = 10;
\end{lstlisting}

\textit{Nota}: JavaScript diferencia variables con el mismo nombre pero distinta cantidad de variables, "nombre" y "Nombre" son dos variables distintas.

Algunas reglas para la declaración de variables en este lenguaje son:
\begin{enumerate}
    \item El primer carácter de una variable debe ser una letra, \textbf{guión bajo} (\_) o un \textbf{símbolo de dolar} (\$).
    \item El primer carácter de una variable no puede ser un número.
    \item Los nombres de variables no pueden incluir operadores matemáticos o lógicos.
    \item Los nombres de variables no pueden contener espacios en blanco.
    \item Los nombres de variables no pueden contener símbolos especiales (", \#, \%, \&, etc).
\end{enumerate}


\subsection{Comentarios}

Para comentar una sola línea de código se utilizan \textbf{dos diagonales} (\textbf{//}) y para comentar múltiples líneas se utiliza los caracteres \textbf{/*} al inicio de las instrucciones que buscas comentar, y \textbf{*/} al final de las instrucciones.
\begin{lstlisting}
    // Esto es un comentario.
    alert("Mensaje dentro de una alerta.")
    /*
    Esto
    También
    Es
    Un
    Comentario.
    */
\end{lstlisting}


\subsection{Tipos de datos}

En este lenguaje, no es necesario declarar una variable con su tipo de dato, sin embargo, no es una buena práctica declarar una variable con un entero, y algunas instrucciones después, asignarle una cadena de caracteres.
\begin{lstlisting}
    // Declaración de variables.
    var x = 1;
    var y = 1.1;
    var z = 1.1111;
    x = "Esto es una variable"; // Esto no es correcto.
\end{lstlisting}

Podemos utilizar una sola comilla (') o dobles comillas (") para contener un texto dentro de una variable, a su vez, podemos utilizar los escapes \textbackslash " y \textbackslash ' para utilizar dichas comillas dentro de una cadena.
\begin{lstlisting}
    var nombre = "mi nombre es \"mario\"";
    var apellido = 'mi nombre es \'casas\''
    var edad = "mi edad es '21'";
\end{lstlisting}

\textit{Nota}: no es necesario utilizar caracteres de escape de una comilla dentro de dobles comillas, ni dobles comillas dentro de comillas.

Algunos caracteres de escape que podemos utilizan se ven en la \textit{Tabla \ref{tab: 1}}:
\begin{table}[H]
    \begin{center}
        \caption{Caracteres de escape válidos}
        \label{tab: 1}
        \begin{tabular}{c l}
            \hline
            \textbf{Carácter de escape}&\textbf{Función} \\
            \hline
            \textbackslash ' & Una comilla \\
            \textbackslash " & Doble comilla \\
            \textbackslash \textbackslash & Diagonal \\
            \textbackslash n & Salto de línea \\
            \textbackslash r & Posiciona el cursor al inicio de la línea \\
            \textbackslash t & Tabulación \\
            \textbackslash b & Posiciona el cursos un carácter atrás en el texto o consola \\
            \textbackslash f & Genera un salto de página \\
            \hline
        \end{tabular}
    \end{center}
\end{table}

Los \textbf{valores booleanos} son: \textbf{true} y \textbf{false}, el primero para casos positivos o reales, el segundo para valores como 0, null, indefinido o cadenas vacías.


\subsection{Operadores}


\subsubsection{Operadores aritméticos}

La \textit{Tabla \ref{tab: 2}} contiene los operadores aritméticos válidos en este lenguaje:
\begin{table}[H]
    \begin{center}
        \caption{Operadores aritméticos en JavaScript}
        \label{tab: 2}
        \begin{tabular}{c l}
            \hline
            \textbf{Operador}&\textbf{Definición} \\
            \hline
            + & Suma o Concatenación \\
            - & Resta \\
            $\ast$ & Multiplicación \\
            / & División \\
            \% & Modulo (residuo de una división) \\
            ++ & Incremento \\
            -- & Decremento \\
            \hline
        \end{tabular}
    \end{center}
\end{table}

La función \textbf{eval()} toma una cadena que contiene una expresión aritmética y regresa su resultado:
\begin{center}
    \textit{console.log(eval("2 + 2")); // Imprime 4.}
\end{center}

Al igual que en otros lenguajes, JavaScript posee los operadores de incremento y decremento post y pre:
\begin{center}
    \textit{
            var++ (incrementa después de una instrucción) \\
            ++var (incrementa antes de una instrucción) \\
            var-- (decrementa después de una instrucción) \\
            --var (decrementa antes de una instrucción) \\
    }
\end{center}


\subsubsection{Operadores de asignación}

La \textit{Tabla \ref{tab: 3}} contiene los operadores de asignación válidos en este lenguaje:
\begin{table}[H]
    \begin{center}
        \caption{Operadores de asignación en JavaScript}
        \label{tab: 3}
        \begin{tabular}{c l}
            \hline
            \textbf{Operador}&\textbf{Equivalencia} \\
            \hline
            = & x = y \\
            += & x = x + y \\
            -= & x = x - y \\
            *= & x = x * y \\
            /= & x = x / y \\
            \%= & x = x \% y \\
            \hline
        \end{tabular}
    \end{center}
\end{table}

\textit{Nota}: pueden combinarse el uso de varios operadores de asignación en una sola instrucción.


\subsubsection{Operadores de comparación}

La \textit{Tabla \ref{tab: 4}} contiene los operadores de comparación válidos en este lenguaje:
\begin{table}[H]
    \begin{center}
        \caption{Operadores de comparación en JavaScript}
        \label{tab: 4}
        \begin{tabular}{c l}
            \hline
            \textbf{Operador}&\textbf{Definición} \\
            \hline
            == & Igual a \\
            === & Idénticos (iguales o del mismo tipo) \\
            != & No igual a \\
            !== & No idéntico \\
            $>$ & Mayor \\
            $>$= & Mayor igual \\
            $<$ & Menor \\
            $<$= & Menor igual \\
            \hline
        \end{tabular}
    \end{center}
\end{table}

Las comparaciones regresan true o false si son ciertas o no.


\subsubsection{Operadores lógicos y booleanos}

La \textit{Tabla \ref{tab: 5}} contiene los operadores lógicos válidos en este lenguaje:
\begin{table}[H]
    \begin{center}
        \caption{Operadores de comparación en JavaScript}
        \label{tab: 5}
        \begin{tabular}{m{3cm} m{10cm}}
            \hline
            \textbf{Operador}&\textbf{Definición} \\
            \hline
            \&\& & Y. Regresa true si ambas expresiones son verdaderas \\
            $||$ & O. Regresa true si una de las expresiones es verdadera \\
            ! & Negación. Regresa el valor contrario (true o false) al resultado de la expresión \\
            \hline
        \end{tabular}
    \end{center}
\end{table}

Este lenguaje soporta el uso del \textbf{operador ternario}:
\begin{lstlisting}
    var mayorDeEdad = (edad < 18) ? "Muy joven" : "Muy viejo";
\end{lstlisting}

Como vimos, está constituido de una condición, el símbolo de pregunta, su primer valor, dos puntos y el segundo valor; solamente soporta dos valores (true y false), a diferencia de una sentencia if, que puede tener varios \textit{if´s anidados} o \textit{else if}.
\begin{lstlisting}
    variable = (condición) ? valor1 : valor2
\end{lstlisting}

\section{Evaluación de expresiones}


\subsection{Sentencia if else}
Evalúa una expresión, si esta es acertada regresa true, si no lo es, regresa false. Su estructura es la siguiente:
\begin{lstlisting}
    if (condición) {
        // Código.
    }
    else {
        // Código.
    }
\end{lstlisting}

\textit{Nota}: el lenguaje soporta que, si solamente se escribe una instrucción en cualquiera de sus bloques, no sea necesario escribir las llaves y el bloque \textit{else} puede ser omitido.


\subsection{Sentencia else if}

En ocasiones, es requerido que se evalúe una condición si una condición previa fue falsa, para eso funciona la sentencia else if:
\begin{lstlisting}
    if (condición 1) {
        // Código.
    }
    else if (condición 2) {
        // Código.
    } else {
        // Código.
    }
\end{lstlisting}


\subsection{Sentencia switch}

Evalúa una variable y se le asigna un valor dependiendo de otra cantidad de valores o parámetros. Su estructura es la siguiente:
\begin{lstlisting}
    switch (variable o expresión) {
        case n1:
            // Código.
            break;
        case n2:
            // Código.
            break;
        .
        .
        .
        case n:
            // Código.
            break;
        default:
            // Código.
    }
\end{lstlisting}

\textit{Nota}: el valor \textbf{default} puede ser omitido y no es obligatoria la palabra reservada \textbf{break}.



\section{Ciclos}

\subsection{Ciclo For}

Ejecuta un bloque de código n cantidad de veces. Su estructura es:
\begin{lstlisting}
    for (inicializador o contador; condición para ejecución; incremento o decremento) {
        // Código.
    }
\end{lstlisting}

Donde:
\begin{itemize}
    \item \textbf{inicializador o contador}: es la variable que será incrementada o decrementada a lo largo de la ejecución del bloque de código. Puede ser omitido siempre y cuando haya alguna variable fuera de la declaración del ciclo que maneje la ejecución del mismo, a su vez, pueden haber múltiples contadores dentro de la declaración.
    \item \textbf{condición para ejecución}: condición que permite que el bloque se ejecute; si la condición ya no es cierta, se sale del ciclo.
    \item \textbf{incremento o decremento}: incrementa o decrementa la variable contador cuando una vuelta se da en el ciclo.
\end{itemize}

Vemos a continuación dos ejemplos del primer punto anterior mencionados:
\begin{lstlisting}
    i = 1;
    // Ciclo for sin un contador.
    for (; i < 5; i++) {}
    // Ciclo for con dos contadores o inicializadores.
    for (x = 1, text = ""; x < 5; x++) {}
\end{lstlisting}


\subsection{Ciclo While}

Ejecuta un bloque de código mientras una condición sea verdadera. Su estructura es:
\begin{lstlisting}
    while (condición) {
        // Código.
    }
\end{lstlisting}

Debe poseer una variable contador dentro de su bloque para manejar las vueltas dentro del ciclo y su fin es que el ciclo termine, sino, sería un ciclo infinito.


\subsection{Ciclo Do-While}

Ejecuta un bloque de código mientras una condición sea verdadera y se ejecuta mínimo una vez. Su estructura es:
\begin{lstlisting}
    do {
        // Código.
    }
    while (condición);
\end{lstlisting}


\subsection{Break y Continue}

La palabra reservada \textbf{break} es utilizada para terminar la ejecución de un ciclo, aún si su condición todavía era verdadera como para continuar con las vueltas.

La palabra reservada \textbf{continue} es utilizada para saltar n cantidad de líneas de código después de la palabra reservada en cuestión, es decir, salta una vuelta del ciclo. Veremos un ejemplo de ambas a continuación:
\begin{lstlisting}
    for (x = 1; x <= 10; x++) {
        if (x == 5){
            continue;
        }
        console.log(x);
        if (x == 8){
            break;
        }
    }

    /*
    Imprime:
    1
    2
    3
    4
    6
    7
    8
    */
\end{lstlisting}

En el código anterior se imprimen los primeros cuatro números, se salta el número cinco por la sentencia \textit{continue} y continua imprimiendo, cuando x vale ocho, termina el ciclo.

\section{Colecciones de datos}


\subsection{Listas}
\hspace{0.55cm}Las \textbf{listas} son los \textbf{vectores}, \textbf{matrices} o \textbf{arreglos} de otros lenguajes de programación, son tipos que almacenan una cantidad finita de elementos de un mismo tipo, pero, la peculiaridad en este lenguaje, es que en una lista puedes almacenar elementos de distintos tipos de datos. Estructura y ejemplos:
\begin{lstlisting}
	<nombre lista> = [<elemento 1>, <elemento 2>, ..., <elemento n> ] #Estructura de declaracion de una lista.

	nombres = ["mario", 'kevin', "jose", 'enrique] #Ejemplo de declaracion de una lista string.
	nums = [0, 9, 1, 8, 7, 6] #Ejemplo de declaracion de una lista int.
	variado = [99, "cien", 9, "kkk"] #Ejemplo de declaracion de una lista con varios tipos de datos.
\end{lstlisting}

Como podemos ver, los ejemplos son declarados de un tipo e inicializados con n cantidad de elementos, para acceder a ellos, trabajarlos o desplegarlos, utilizamos el \textbf{operador[]}, cada elemento posee un \textbf{índice}, que va de 0 a n, entonces, desplegaremos como ejemplo un elemento de una de las listas anteriores en el siguiente ejemplo:
\begin{center}
	\textit{print(nombres[1]) //Despliega "kevin".\\print(nums[0]) //Despliega 0.\\print(variado[3]) //Despliega "kkk".}
\end{center}

Podemos hacer que los elementos de una lista sean listas también, agregando una dimensión adicional a la lista (como los arreglos multidimensionales), como se ve a continuación:
\begin{lstlisting}
	nums = [[1, 2, 3], [4, 5, 6], [7, 8, 9]]
	n = [0, [1, 2, 3], 9, [2, 2, 2], 1, 3, 5, [6, 7, 8]] #Ejemplo de declaracion de una lista int.
	
	print(nums[1][1]) #Imprime 5.
	print(n[0]) #Imprime 0.
	print(n[1][0]) #Imprime 1.
\end{lstlisting}

Apreciamos que, para acceder a un elemento multidimensional de una lista, existen n pares de llaves cuadradas ([]), donde van ingresados los índices del elemento al que queremos acceder. Cuando son multidimensional las listas, podemos ver las listas de dos dimensiones como tablas, con filas y columnas, donde Python primero recorre las filas y luego las columnas para posicionarse en el elemento con índice solicitado; cuando la lista tiene elementos listas y elementos que no son listas, podemos ver que al elemento que no es lista solo se le aplica el uso de un par de llaves cuadradas, mientras a los elementos lista si se les aplica el uso de varios pares de llaves cuadradas.

Podemos aplicar el uso del \textit{operador[]} a las cadenas. Si tenemos una variable cadena y utilizamos el par de llaves cuadradas con un número entero como índice, accedemos al caracter en dicho índice de la cadena, como vemos en el siguiente ejemplo:
\begin{center}
	\textit{nombre = 'Luis' //Declara e inicializa cadena con nombre.\\print(nombre[1]) //Despliega "u".}
\end{center}

También podemos acceder a ciertos elementos de la lista, utilizando los \textbf{dos puntos (:)} dentro de los paréntesis cuadrados del nombre de la lista:
\begin{lstlisting}
	nums = [1, 3, 5, 7, 9, 11]
	
	print(nums[0:1]) #Imprime 1.
	print(n[1:4]) #Imprime 3, 5, 7.
	print(n[4:5]) #Imprime 11.
\end{lstlisting}

Puede pasar que olvidemos escribir un número en alguno de los dos lados de los dos puntos, sin embargo, esto no es un error, podemos obtener los elementos partiendo de uno de los dos numeros escritos del lado izquierdo o derecho de los dos puntos, además, podemos obtenerlos de uno en uno, dos en dos, tres en tres, etc:
\begin{lstlisting}
	nums = [1, 3, 5, 7, 9, 11]
	
	print(n[:3]) #Imprime 1, 3, 5.
	print(n[1:]) #Imprime 3, 5, 7, 9, 11.
	print(n[1:5:2]) #Imprime 3, 7.
	print(n[1::2]) #Igual que la instruccion anterior.
\end{lstlisting}

A su vez, podemos obtener los elementos de una lista de manera negativa, es decir, de n elemento hacía atrás:
\begin{lstlisting}
	nums = [1, 3, 5, 7, 9, 11]
	
	print(n[:4:-1]) #Imprime 7, 5, 3, 1.
	print(n[::-1]) #Imprime 11, 9, 7, 5, 3, 1.
\end{lstlisting}


\subsubsection{Operaciones con listas}
\hspace{0.55cm}Puedes realizar distintas operaciones similares a las que se pueden hacer con cadenas y con otras expresiones, como vemos en la \textit{Tabla \ref{tab: 7}}:
\begin{table}[H]
    \begin{center}
        \caption{Operaciones con listas}
        \label{tab: 7}
        \begin{tabular}{m{4cm} m{5cm} m{4cm}}
            \hline
            \textbf{Operación} & \textbf{Significado} & \textbf{Ejemplo} \\
            \hline
            Asignación de valor a un elemento en concreto		& Puedes acceder a un elemento de la lista y cambiar su valor		& n = [1, 3, 5] n[0] = 33 print(n) \\
            Sumas una lista a otra								& Puedes agregarle una lista a otra con el operador +				& n = [1, 3, 5] n += [7, 9, 11] print(n) \\
            Operaciones aritméticas entre elementos			& Puedes sumar, multiplicar, restar, dividir u obtener el residuo de una división entre elementos de una misma lista o distintas							& n = [1, 3, 5] m = [7, 9, 11] print(n[1] // m[1]) \\
            Multiplicación de listas							& Repite todos los elementos de una lista n cantidad de veces		& n = [1, 3, 5] print(n * 4) \\
            Búsqueda de un elemento								& Se busca un elemento en l alista con la palabra reservada \textbf{in}, regresa True si lo halló, en caso contrario, regresa False						& n = [1, 3, 5] if (7 in n): print("Encontrado") else: print("No encontrado") print(1 in n) \\
            Comprobar si un elemento mo está en la lista			& Se comprueba si un elemento no existe en una lista con la palabra reservada \textbf{not}, regresa True si no existe, en caso contrario, regresa False	& n = [1, 3, 5] if (7 not n): print("No existe") else: print("Existe") print(1 not n) \\
            \hline
        \end{tabular}
    \end{center}    
\end{table}


\subsubsection{Funciones de listas}
\hspace{0.55cm}La \textit{Tabla \ref{tab: 8}} contiene algunas de las funciones más útiles de las listas son:
\begin{table}[H]
    \begin{center}
        \caption{Funciones especiales para listas}
        \label{tab: 8}
        \begin{tabular}{m{4cm} m{5cm} m{4cm}}
            \hline
            \textbf{Función}&\textbf{Significado}&\textbf{Ejemplo} \\
            \hline
            \textbf{len(lista)}					& Regresa la cantidad de elementos										& len(lista) \\
            \textbf{append(elemento)}			& Agrega un elemento al final de la lista								& lista.append(4) \\
            \textbf{insert(índice, elemento)}	& Agrega un elemento en el índice deseado 								& lista.insert(0, 10) \\
            \textbf{index(elemento)}			& Regresa el índice de la primer aparición de un elemento en la lista	& letras.index("r") \\
            \textbf{max(lista)}					& Regresa el valor más alto de la lista									& max(numeros) \\
            \textbf{min(lista)}					& Regresa el valor más pequeño de la lista								& min(numeros) \\
            \textbf{count(elemento)}			& Regresa la cantidad de veces que se repite un elemento en la lista	& letras.count("r") \\
            \textbf{remove(elemento)}			& Remueve un elemento existente de la lista								& letras.remove("r") \\
            \textbf{reverse()}					& Invierte los elementos de la lista										& numeros.reverse() \\
            \hline
        \end{tabular}
    \end{center}    
\end{table}


\subsubsection{Listas con reglas}
\hspace{0.55cm}Podemos declarar elementos de una lista siguiendo una condicional o un ciclo, por ejemplo, declarar los elementos de una lista que estén elevados a la tercera potencia, o que sean números pares, etc. Esto se consigue metiendo ciclos for o condicionales dentro de los paréntesis cuadrados [], como vemos ahora:
\begin{lstlisting}
   #Declara lista con cinco elementos a la tercera potencia.
   cubos = [i ++ 3 for i in range(5)]
   
   print(cubos) #Imprime [0, 1, 8, 27, 64].
   
   #Declara lista con números al cuadrado y que sean pares.
   pares = [i ** 2 for i in range(10) if i ** 2 % 2 == 0]
   
   print(pares) #Imprime [0, 4, 16, 36, 64]
\end{lstlisting}


\subsubsection{Ordenamiento de listas}
\hspace{0.55cm}La forma nativa para ordenar una lista en Python es por medio de la función de listas \textbf{sort()}:
\begin{lstlisting}
	# Declara lista con números
	numeros = [16, 4, 9, 1, 3, 20, 8]
	# Llamada a la función que ordena ascendentemente una lista.
	numeros.sort()
	# Imprime la lista.
	print(numeros)
	
	# Imprime: [1, 3, 4, 8, 9, 16, 20]
\end{lstlisting}

Esta función cambia a la lista propiamente, si queremos crear una lista ordenada a partir de una desordenada, podemos utilizar la función \textbf{sorted()}:
\begin{lstlisting}
	# Declara lista con números
	numeros = [16, 4, 9, 1, 3, 20, 8]
	# Asigna a una nueva lista la llamada a la función que ordena ascendentemente una lista.
	nums = numeros.sorted()
	# Imprime las lista.
	print(numeros)
	print(nums)
	
	# Imprime:
	# [16, 4, 9, 1, 3, 20, 8]
	# [1, 3, 4, 8, 9, 16, 20]
\end{lstlisting}

Por defecto en ambas funciones, las listas son ordenadas de manera ascendente, sin embargo, los dos métodos poseen un parámetro llamado \textbf{reverse}, el cual está inicializado en \textit{false}, si cambiamos su valor a \textit{true}, la lista será ordenada descendentemente:
\begin{lstlisting}
	# Declara lista con números
	numeros = [16, 4, 9, 1, 3, 20, 8]
	# Llamada a la función que ordena descendentemente una lista.
	numeros.sort(reverse=true)
	# Imprime la lista.
	print(numeros)
	
	# Imprime: [20, 16, 9, 8, 4, 3, 1]
\end{lstlisting}

Si se quiere ordenar una lista irregular (que contenta cadenas, carácteres, números o decimales), las función sort y sorted lanzarán un error.


\subsection{Pilas}
\hspace{0.55cm}Las \textbf{pilas} son una estructura de datos que sigue la filosofia de Último en llegar, Primer en irse (LIFO, Last In, First Out), estas pueden ser estáticas (no cambia tope de elementos) o dinámicas (cambia su tope de elementos durante ejecución).

Python no suele ser estático, por lo cual, las pilas tampoco, se les puede establecer un número tope de elementos a almacenar, pero se mantiene su escencia dinámica; en otros lenguajes, existen librerias o clases que deben ser importadas o adjuntadas a nuestro proyecto para trabajar con métodos especializados para pilas, sin embargo, este lenguaje no posee algo similar, las \textbf{listas} son pilas en sí, así que nos limitaremos simplemente a crear una clase con las funciones básicas de una pila:
\begin{lstlisting}
	class Pila:
		def __init__(self):
			self.pila = []
			
		def agregar_elemento(self, elemento):
			self.pila.append(elemento)
			
		def eliminar_elemento(self. elemento):
			self.pila.pop()
			
		def verificar_vacia(self):
			return self.pila == []
			
		def buscar_elemento(self, elemento):
			if elemento in self.pila:
				return true
			else
				return false
				
		def desplegar_elementos(self):
			print(self.pila)
			
		def vaciar_pila(self):
			self.pila.clear()
\end{lstlisting}

La clase posee un método para agregar, eliminar, recorrer, buscar y vaciar la pila, además de verificar si está vacía o no, esto gracias a las funciones de las listas. La lista actua como una pila debido a que la función \textbf{append()} agrega los elementos al final de la lista (se agrega el primero, se agrega el segundo después y el primero se recorre, entonces el segundo ahora es el primero y así sucesivamente), y la función \textbf{pop()} elimina el último elemento de la lista, respetando así que el último elemento entrar es el primero en salir o ser eliminado.


\subsection{Colas}
\hspace{0.55cm}Las \textbf{pilas} son una estructura de datos que sigue la filosofia de Primero en llegar, Primer en irse (FIFO, First In, First Out), estas pueden ser estáticas (no cambia tope de elementos) o dinámicas (cambia su tope de elementos durante ejecución).

A diferencia de las pilas, esta estructura si cuenta con un módulo que podemos importar a Python, la cual es llamada \textbf{deque}:
\begin{center}
	\textit{from collections import deque}
\end{center}

Dejaremos las operaciones básicas que se pueden realizar con las colas y su módulo importado en una clase:
\begin{lstlisting}
	from collections import deque

	class Cola:
		def __init__(self, cola):
			self.cola = []
			
		def agregar_elemento(self, elemento):
			self.cola.append(elemento)
			
		def eliminar_elemento(self. elemento):
			self.cola.popleft()
			
		def verificar_vacia(self):
			return self.cola == []
			
		def buscar_elemento(self, elemento):
			return elemento in self.cola
				
		def desplegar_elementos(self):
			print(self.cola)
			
		def vaciar_cola(self):
			self.cola.clear()
\end{lstlisting}

\textit{Nota}: fíjese que el método para eliminar un elemento de la cola es llamado \textbf{popleft()}, esto es así debido a que el módulo \textbf{deque} tiene dos métodos para eliminar elementos: el anterior mencionado y \textbf{popright()}.

Para poder trabajar con esta clase y el módulo \textit{deque}, primero debemos crear un objeto del módulo anterior y, posterior a ello, crear un objeto de la clase que nosotros creamos, y pasarle como parámetro el objeto \textit{deque}; podríamos trabajar tranquilamente solamente con el objeto deque y sus métodos mostrados en el ejemplo anterior, pero para mayor comodidad de lectura, decidimos implementarlo dentro de una clase:
\begin{lstlisting}
	# Si se imprime la siguiente línea, resulta en una lista vacía.
	nueva_cola = deque()
	
	clase_cola = Cola(nueva_cola)
\end{lstlisting}


\subsection{Rangos}
\hspace{0.55cm}Los \textbf{rangos} son una lista con una secuencia de números dada, si requerimos una lista con los números del 1 al 100, podemos utilizar esta herramienta, \textbf{no regresa el último número} (si queremos un rango de 1 al 100, obtenemos el rango del 0 al 99, ya sabemso que muchos lenguajes de programación utilizan el 0 como primer valor de iteración). Un ejemplo de los rangos con tres tipos de casos.
\begin{lstlisting}
	numeros = range(100) #Genera un rango de 100 numeros (0 - 99).
	x = range(1, 20) #Genera un rango de uno a 19 (1 - 19).
	y = range(1, 30, 3) #Genera un rango de uno a 30 de tres en tres (1, 4, 7, 10, ..., 28).
	z = range(100, 1, -5) #Genera un rango negativo de 100 a uno de menos cinco en menos cinco (100, 95, 90, 85, ..., 5).
	
	for a in range(5):
		print("hola mundo") #Imprime cinco veces el mensaje "hola mundo".
\end{lstlisting}

Como vimos, la función range puede crear rangos negativos, que vaya de n en n números, y pueden ser utilizados para repetir un ciclo for n cantidad de veces.

También podemos convertir los rangos a listas, con la función \textbf{list()}.


\subsection{Diccionarios}
\hspace{0.55cm}Los \textbf{diccionarios} son un tipo de colección de datos donde cada \textbf{llave} (identificador) tiene un \textbf{valor} (contenido), algo así como una dupla, pero los diccionarios son colecciones de duplas; podemos acceder a los valores de los diccionarios utilizando el nombre del diccionario y su llave.
\begin{lstlisting}
   Nombres = { #Declara diccionario.
      //Inicializa diccionario con tres llaves y valores.
      "David" : 19
      "Maria" : 26
      "Luis" : 21
   }
   
   print(Nombres["David"]) #Despliega el valor de la llave (19).
\end{lstlisting}

Vemos que se utilizan tipos de datos básicos para las llaves y valores, estos tipos son \textbf{mutables}, quiere decir que puede cambiar de valor o ser modificado, a la par de que puede cambiar de tipo de dato (recordemos que esto no es una buena practica en el diseño de programas, es posible, pero no recomendable); las \textit{listas} también son mutables, por lo que podemos utilizar listas como valores de una llave, no como una llave en sí, si utilizamos una lista como una llave, causaría un error.


\subsubsection{Funciones en diccionarios}
\hspace{0.55cm}Podemos utilizar las palabras reservadas \textbf{in} o \textbf{not in} para determinar si una llave existe o no en un diccionario; usamos \textbf{get} para desplegar el valor de una llave, algo como el operador [] para acceder al valor, sin embargo, si no se encuentra la llave en el diccionario, despliega otro valor como resultado. Veamos el siguiente ejemplo:
\begin{lstlisting}
   #Declara diccionario.
   Numeros = {
      #Inicializa diccionario con llaves y valores.
      1 : "UNO"
      2 : "DOS"
      3 : "TRES"
      4 : "CUATRO"
      5 : "CINCO"
      6 : "SEIS"
      7 : "SIETE"
   }
   
   #Si no existe la llave en numeros, despliega mensaje.
   if (8 not in Numeros):
      print("No existe esa llave.")
   #Si 1 existe en mensaje, despliega su valor.
   elif (1 in Numeros):
      print(Numeros[1])
      #Busca la llave 2 y despliega su valor, sino, despliega "Ese es un dos".
      print(Numeros.get(2, "Ese es un dos"))
\end{lstlisting}


\subsection{Tuplas}
\hspace{0.55cm}Las \textbf{tuplas} son como las listas, pero \textbf{inmutables} (no puedes cambiar el valor de uno de sus elementos, si lo intentas, ocurrirá un error), se inicializa con paréntesis (o sin los paréntesis) y se pueden acceder a sus elementos por medio del operador [], como vemos en el siguiente ejemplo:
\begin{lstlisting}
   #Declara diccionario.
   Nombres = {
      "David" : 19
      "Maria" : 26
      "Luis" : 21
   }
   
   #Declara lista.
   Numeros = [1, 2, 3, 4, 5]
   
   #Declara tupla con parentesis.
   Municipios = ("Tijuana", "Mexicali", "Tecate", "Ensenada")
   
   #Declara tupla sin parentesis.
   Estados = "Baja California", "Baja California Sur", "Sonora", "Chihuahua"
   
   print(Municipio[2]) #Despliega Tecate.
   
   Estados[0] = "Quintana Roo" #Error.
\end{lstlisting}

Vemos que las listas son declaradas con paréntesis cuadrados [], los diccionarios con llaves \{\}, y las tuplas con paréntesis ().

Pueden ser utilizadas las tuplas en las listas, si recordamos, mencionamos previamente que los diccionarios pueden contener tipos de datos mutables, no inmutables para las llaves, sin embargo, podemos crear un diccionario y que sus elementos sean tuplas, esto da como resultado un diccionario con valores que no pueden ser modificados, logrando así mantener la integridad de los datos.


\subsubsection{Desempacando tuplas}
\hspace{0.55cm}Si tenemos un conjunto de variables y una tupla con la misma cantidad de elementos dentro de ella, podemos asignar cada uno de los elementos a cada una de las variables, a esta acción se le llama \textbf{desempacar tuplas}, además, podemos asignar todos los elementos a la derecha de una tupla a una sola variable, con el operador *, como vemos a continuación:
\begin{lstlisting}
   #Declara tupla.
   Municipios = ("Tijuana", "Mexicali", "Tecate", "Ensenada")
   
   #Declara tupla.
   Estados = "Baja California", "Baja California Sur", "Sonora", "Chihuahua"
   
   #A a se le asigna Tijuana, a b se le asigna Mexicali,
   #a c se le asigna Tecate, y a d se le asigna Ensenada.
   a, b, c, d = Municipios
   
   #A e se le asigna Baja California, y a f se le asigna todos los elementos
   #restantes de la tupla.   
   e, *f = Estados   
   
   #Imprime las variables.
   print(a)
   print(b)
   print(c)
   print(d)
   print(e)
   print(f)
\end{lstlisting}


\subsection{Sets}
\hspace{0.55cm}Los \textbf{conjuntos} (o \textbf{sets}, como son conocidos), son algo así como una lista o diccionario, pero, a diferencia de las listas, los elementos de un conjunto no deben estar duplicados, a la vez que están desordenados, por lo que no se pueden acceder por medio del operador [], logrando con esto que la búsqueda de los elementos sea más rápida que con las listas.

Si un elemento del conjunto está duplicado, a la hora de imprimirlo, solo se muestra una vez, si se elimina uno de los elementos duplicados, se eliminan el resto de las copias, y si se imprime la cantidad de elementos, se descartan todos los elementos duplicados (si se repite cuatro veces 1, solo se imprime una vez 1).

Se declaran igual que una lista, podemos utilizar los métodos \textbf{add(elemento)},  \textbf{remove(elemento)} y \textbf{len()} para agregar, eliminar un elemento respectivamente y desplegar la cantidad de items del conjunto, como vemos a continuación:
\begin{lstlisting}
   #Declara conjunto con elementos duplicados.
   numeros = {1, 1, 3, 4, 5, 7, 8, 8, 9}
   
   print(numeros) #Imprime 1, 3, 4, 5, 6, 7, 8, 9.
   print(len(numeros)) #Imprime 7.
   
   numeros.add(0) #Agrega 0 al conjunto.
   numeros.remove(1) #Elimina todos los 1 del conjunto.
   
   print(numeros) #Imprime 0, 3, 4, 5, 6, 7, 8, 9.
   print(len(numeros)) #Imprime 7.
\end{lstlisting}


\subsubsection{Operaciones con conjuntos}
\hspace{0.55cm}Como en las matemáticas, podemos aplicar las operaciones de conjuntos a esta colección de datos (\textit{Tabla \ref{tab: 9}}):
\begin{table}[H]
    \begin{center}
        \caption{Operaciones para conjuntos}
        \label{tab: 9}
        \begin{tabular}{m{4cm} m{5cm} m{4cm}}
            \hline
            \textbf{Operación}&\textbf{Significado}&\textbf{Ejemplo} \\
            \hline
            \textbf{union (|)}					& Regresa la suma de todos los elementos de los conjuntos							& primero | segundo \\
            \textbf{intersection (\&)}			& Regresa solamente los elementos que coinciden en ambos conjuntos					& primero \& segundo \\
            \textbf{difference (-)}			& Regresa los elementos de un conjunto, pero no del otro, ni los que coinciden	& primero - segundo \\
            \textbf{symmetric difference (\^)}	& Regresa los elementos que no coinciden en ambos conjuntos						& primero \^ segundo \\
            \hline
        \end{tabular}
    \end{center}    
\end{table}

El resultado de los ejemplos anteriores se puede ver a continuación:
\begin{lstlisting}
   #Declara dos conjuntos.
   first = {1, 2, 3, 4, 5, 6}
   second = {4, 5, 6, 7, 8, 9}

   print(first | second) #Imprime {1, 2, 3, 4, 5, 6, 7, 8, 9}
   print(first & second) #Imprime {4, 5, 6}
   print(first - second) #Imprime {1, 2, 3}
   print(second - first) #Imprime {8, 9, 7}
   print(first ^ second) #Imprime {1, 2, 3, 7, 8, 9}
\end{lstlisting}


\subsection{Seleccionando una colección de datos}
\hspace{0.55cm}La siguiente lista da algunos tips sobre como escoger una colección de datos que más se adecue a nuestras necesidades:
\begin{itemize}
	\item Si necesitas dos valores que vayan ligados o tengan una relación directa, velocidad a la hora de buscar valores y, que a su vez, estos valores puedan cambiar, utiliza un \textbf{diccionario}.
	\item Si necesitas un conjunto de datos sin orden y sin relación con otros datos y, que a su vez, sea bastante sencillo e iterable, utiliza una \textbf{lista}.
	\item Si requieres de un conjunto de datos que no estén repetidos, utiliza un \textbf{conjunto}.
	\item Si requieres un conjunto de datos que no puedan ser modificados, utiliza una \textbf{tupla}.
	\item Las \textit{tuplas} pueden ser combinadas con el resto de colecciones de datos.
\end{itemize}
\section{Funciones}
\hspace{0.55cm}Una \textbf{función} es un bloque de código que puede ser llamado dentro de otra sección del código para realizar una tarea en específico. Las ventajas de las funciones es que puede crear las tuyas que hagan una determinada tarea que requieras, además de que estas funciones son reutilizables y puedes probarlas cada una individualmente.

Toda función tiene la siguiente estructura:
\begin{lstlisting}
    [Tipo de dato] [Nombre] ([Parámetro1], [Parámetro2], ..., [ParámetroN])
    {
        //Código
        return [Variable o expresión del mismo tipo de dato de la función]
    }
\end{lstlisting}

Los parámetros no son precisamente obligatorios, se ponen para fines de establecer la estructura de una función, lo que si es requerido es que la instrucción \textbf{return} regrese un resultado, variable o expresión del mismo tipo que sea la función, sino había un error.

En ocasiones, las funciones no regresarían un valor, simplemente cumplen con una tarea, es decir, regresan un vacío, a este tipo de funciones se le llaman \textbf{procedimientos} y su estructura es la siguiente:
\begin{lstlisting}
    void [Tipo de dato] [Nombre] ([Parámetro1], [Parámetro2], ..., [ParámetroN])
    {
        //Código
    }
\end{lstlisting}

Para llamar y ejecutar una función o procedimiento se tiene que haber creado previo a su utilización, debemos simplemente escribir el nombre de la función o procedimiento, seguido del inicio y cierre de paréntesis, y punto y coma.
\begin{lstlisting}
    void Despliegue()
    {
        cout << "Hola mundo";
    }
    
    int main()
    {
        Despliegue(); //Despliega "Hola mundo";
        
        return 0;
    }
\end{lstlisting}

Una \textbf{función prototipo} consiste en la declaración simple de una función o procedimiento, pero sin darle un bloque de código a ejecutar, es como declarar una variable y no inicializarla, con esto, podemos crear \textit{funciones prototipo}, llamarlas en el \textit{main}, y darles un cuerpo o bloque a dichas funciones por debajo de main o en otra ubicación que deseemos.
\begin{lstlisting}
    void Despliegue1(); //Prototipo de procedimientos
    void Despliegue2();
    
    int main()
    {
        Despliegue1(); //Despliega "Hola mundo"
        Despliegue2(); //Despliega "Hola mundo"
        
        return 0;
    }
    
    void Despliegue1()
    {
        cout << "Hola mundo1";
    }
    
    void Despliegue2()
    {
        cout << "Hola mundo2";
    }
\end{lstlisting}


\subsection{Con parámetros}
\hspace{0.55cm}Si somos más específicos, una función puede tener \textbf{argumentos}, y los valores que le pasamos a dichos argumentos se llaman \textbf{parámetros}, entonces, para que una función o procedimiento tenga argumentos, debemos seguir la siguiente estructura:
\begin{lstlisting}
    void Despliegue([tipo de dato] [nombre])
    {
        //Código
    }
\end{lstlisting}

El siguiente ejemplo muestra un procedimiento que acepta un argumento y, en su llamada, le pasamos un parámetro:
\begin{lstlisting}
    void Despliegue(string mensaje)
    {
        cout << mensaje;
    }
    
    int main()
    {
        string mensaje = "que tal";
        Despliegue(mensaje); //Despliega "que tal";
        
        return 0;
    }
\end{lstlisting}

Si bien un \textit{procedimiento} no regresa un valor y realiza una tarea sin más, una función si regresa un valor, y este valor puede ser asignado a una variable, como si se tratase de una asignación cualquiera, siempre y cuando sean del mismo tipo de dato. Además, la instrucción \textit{return} puede modificar el argumento que recibe la función para devolverlo alterado:
\begin{lstlisting}
    int Multiplicacion(int x)
    {
        return x*10;
    }
    
    int main()
    {
        int numero;
        numero = Multiplicacion(10); //La función le asigna a la variable el valor 10 multiplicado por 10
        
        return 0;
    }
\end{lstlisting}


\subsubsection{Argumentos predeterminados}
\hspace{0.55cm}Una función o procedimiento puede argumentos con \textbf{valores predeterminados}, esto quiere decir que, si le hablamos a la función o procedimiento, y a esta no le pasamos ningún parámetro que corresponda con su argumento en la estructura de la función, esta misma asumirá que dicho argumento tiene asignado su valor por defecto, esto aplica para funciones y procedimientos con n argumentos.
\begin{lstlisting}
    int Suma(int x, int y, int z = 5) //El valor predeterminado del argumento z es 5
    {
        return x + y + z;
    }
    
    int main()
    {
        int resultado1 = Suma (1, 2, 3); //Llama a Suma y le pasa tres parámetros
        int resultado2 = Suma(2, 2); //Llama a Suma y le pasa dos parámetros
        
        cout << "Resultado  1: " << resultado1 << endl; //Despliega 6
        cout << "Resultado  2: " << resultado2 << endl; // Despliega 9
    }
\end{lstlisting}


\subsection{Con múltiples parámetros}
\hspace{0.55cm}Le puedes poner tantos argumentos como uno desee, separados por comas , siempre y cuando cada argumento tenga su tipo de dato y nombre. Para llamar a la función o procedimiento, se sigue la misma lógica que tiene el llamar una función con un solo parámetro, si tiene varios, se separan por una coma. 
\begin{lstlisting}
    int Multiplicacion(int x, int y, int z)
    {
        return x+y+z;
    }
    
    int main()
    {
        int numero;
        numero = Multiplicacion(10, 20, 30); //La función le asigna a la variable el valor 10 más 20 más 30
        cout << numero;
        
        return 0;
    }
\end{lstlisting}


\subsection{Arreglos como parámetros}
\hspace{0.55cm}Una función o procedimiento también \textbf{arreglos} como argumento, basta con indicarle al argumento un tipo de dato, un nombre identificativo y los corchetes (sin ningún contenido dentro de ellos).
\begin{center}
    \textit{void DespliegueArreglo(int arreglo[], int size) \{\}}
\end{center}

Para llamar a la función o procedimiento y pasarle como parámetro un arreglo, debemos escribir (en su correspondiente argumento) el nombre del arreglo sin los corchetes, C++ reconoce que es un arreglo y asume todos sus valores sin poner los corchetes. A continuación se presenta un ejemplo en código:
\begin{lstlisting}
    void DesplegarArreglo(int arreglo[], int size)
    {
        int i;
        
        for (i = 0; i < size; i++)
        {
            cout << arreglo[i] << endl;
        }
    }
    
    int main()
    {
        int tam, i;
        
        cout << "Tamaño del arreglo: ";
        cin >> tam;
        
        int arr[tam];
        
        for (i = 0; i < tam; i++)
        {
            arreglo[i] = i;
        }
        
        DesplegarArreglo(arr, tam);
    }
\end{lstlisting}


\subsection{Punteros como referencias}
\hspace{0.55cm}Por defecto, C++ tiene por defecto el pase de valores por parámetro hacías las funciones y procedimientos, pero existen otra forma de pasarle un parámetro a una función o procedimiento:
\begin{itemize}
    \item \textbf{Por valor}: lo que pasa cuando se le manda valor por parámetro a la función o procedimiento es que, desde donde se llama, se copia el valor del argumento (variable) local y dicha copia se le pasa como parámetro a la función o procedimiento, por lo que ambos quedan totalmente separados y dentro de la función, se le puede hacer lo que quiera al valor obtenido por parámetro (es como una copia de un archivo, el orinal queda intacto, mientras que la copia la puedes modificar como desees).
    \item \textbf{Por referencia}: contrario a lo expuesto anteriormente, en vez de realizar una copia del argumento (variable) local, se crea una referencia de él la cual pasa como parámetro a la función o procedimiento, dentro de esta, lo que se le haga a la referencia afectará a su origen (es como el acceso directo a un archivo).
\end{itemize}

El pase por valor ya lo conocemos, es que usualmente se utiliza. El \textbf{pase por referencia funciona} debido a que esta referencia es la dirección de memoria del argumento que se le pase a la función o procedimiento por parámetro, recordemos que una dirección de una variable contiene su valor también, es por ello que, cuando se modifica un valor pasado por referencia dentro de la función, este también se ve modificado desde su origen.

Creando un procedimiento que acepte parámetros por referencia, debemos utilizar \textit{punteros} para ello, estos deben estar indicados en la estructura del procedimiento o función, con su tipo de dato y el operador * (que indica la declaración de un puntero). Cuando llamamos a la función, simplemente le pasamos la dirección de la variable a la que apunta el puntero (con el símbolo \textbf{\&}). Refresquemos la memoria y lo leído con el siguiente ejemplo:
\begin{lstlisting}
    void Funcion(int *x) //El parámetro de esta función es un puntero llamado x del tipo int
    {
        *x = 100; //El * antes del nombre del puntero establece el valor del mismo
    }
    
    int main()
    {
        int variable = 25; //Variable local del tipo int
        Funcion(&variable); //Le pasamos la referencia de variable local, la función le asigna el valor de 100 y después se despliega
        cout << variable;
    
        return 0;
    }
\end{lstlisting}

De manera general, el uso de parámetros por valor es más efectivo, rápido y requiere menos memoria, a diferencia del pase por referencia, además, evitamos tener que pensar en alterar el valor original de la variable que pasó su valor por referencia con el pase por valor.


\subsection{Sobrecarga de funciones}
\hspace{0.55cm}\textbf{Sobrecargar una función o procedimiento} permite duplicar dichas funciones y procedimientos, pero con distintos parámetros, para ejecutar las mismas tareas pero con ligeras variaciones. Es preferible que estos procedimientos sobrecargados se diferencien por medio de distintos tipos de datos para los argumentos, la cantidad de los mismos, y la adición o recorte de alguna instrucción en su bloque de código.
\begin{lstlisting}
    void Suma(int x, int y)
    {
        int resultado = x + y;
        cout << resultado;
    }
    
    void Suma(float x, float y)
    {
        float resultado = x + y;
        cout << resultado;
    }

    int main()
    {
        Suma(1, 2);
        Suma (3.14, 5.66);
    }
\end{lstlisting}

Un error casual a la hora de ejecutar esto, es que se quieran sobrecargar funciones solamente cambiando su tipo de dato que devuelve, pero manteniendo tipos de argumentos y cantidad de los mismos similares.
\begin{lstlisting}
    int Despliegue (int a) { }
    float Despliegue (int b) { }
    double Despliegue (int c) { }
\end{lstlisting}


\subsection{La función rand()}
\hspace{0.55cm}Generar números aleatorios nos permite probar distintas funciones o códigos, C++ requiere de la cabecera \textbf{cstdlib} para poder acceder a la función \textbf{rand()}.
\begin{center}
    \textit{\#include $<$cstdlib$>$}
\end{center}

Algo que se debe de saber es que estos números aleatorios son pseudoaleatorios, esto quiere decir, en C++, una vez se termine la ejecución y se vuelva a ejecutar, los números que anteriormente aparecieron, volverán a aparecer, además. estos números son enteros.

Podemos hacer que los números generados por \textit{rand()} estén contenidos dentro de un rango, para ello, utilizamos el operador \textbf{modulo (\%)} y la siguiente estructura:
\begin{lstlisting}
    #include <cstdlib> //Cabecera para utilizar la función rand()
    
    int main()
    {
        int numero = rand(); //Se le asigna a la variable un número aleatorio
        int numerorango = 1 + (rand() % 6); //Se le asigna a la variable un número aleatorio entre el 1 y el 6
        
        cout << numero << ", " << numero rango << endl;
    }
\end{lstlisting}


\subsubsection{La función srand()}
\hspace{0.55cm}Esta función te permite \textbf{generar auténticos} números aleatorios. Esta función permite que se le pase un valor especial como parámetro, el cual es utilizado por la función \textit{rand()}, este valor es un tipo de "fuente", de la cual rand toma valores aleatorios; si se pone un número fijo, realmente no pasa nada y no se generan números aleatorios auténticos, en cambio, si ponemos una fuente que esté constantemente cambiando su valor, podríamos utilizar y considerar esto como valores aleatorios reales, por ejemplo, la función \textbf{time()}, el siguiente ejemplo explica lo escrito:
ç\begin{lstlisting}
    #include <ctime> //Cabecera que acepta el tiempo actual del sistema

    int main()
    {
        srand(time(0)); //El tiempo actual del sistema con parámetro 0 (segundos), es el valor especial del cual rand() obtendrá sus valores
        
        for (int i = 1; i <= 10; i++)
        {
            cout << 1 + (rand() % 6) << endl;
        }
        
        return 0;
    }
\end{lstlisting}


\subsection{Funciones recursivas}
\hspace{0.55cm}Una \textbf{función recursiva} es aquella que se llama a sí misma, al igual que con los \textit{ciclos while}, se debe escribir una condición o expresión de salida, para evitar un recursión indefinida.

La condición o expresión de salida suele llamarse \textbf{el caso base}, el cual hace que se evite una recursión infinita. Existen muchos ejemplos donde podemos utilizar estas funciones, como lo puede ser la famosa función matemática \textbf{factorial}, ejemplificaremos las funciones recursivas con este caso:
\begin{lstlisting}
    int factorial(int n)
    {
        if (n == 1)
            return 1;
        else
            return n * factorial(n - 1);
    }
    
    int main()
    {
        cout << factorial(5); //Despliega 5 * 4, 4 * 3, 3 * 2, 2 * 1, 1
    }
\end{lstlisting}
\section{Clases}

Creamos una clase con la palabra reservada \textbf{class}, seguido de su nombre y dos puntos (:).

Para crear \textbf{métodos} de una clase, sabiendo que siguen la sintaxis de una función,  agregamos indentación a la clase, y le damos un nombre y parámetros. Todos los métodos creados dentro de una clase deben tener un parámetro llamado \textbf{self}, refiriéndose a algún atributo de la clase.

Para crear \textbf{atributos} de una clase no es necesario crear una sección con encapsulamiento, ni tipos de datos, ni nombres, como dijimos previamente, basta con utilizar la palabra reservada \textit{self} en cada método creado (ya sea el constructor u otro alguno) para que la clase sepa que dicho parámetro es un atributo de la clase, podemos acceder a él desde un objeto de clase por medio del operador ., simplemente recuerde qué atributos tendrá la clase y puede apuntarlos en comentarios, para recordarlos más fácilmente.


\subsection{Constructores}

Este es el método llamado \textbf{\_\_ init\_\_}, el cual es llamado cuando un objeto de una clase es llamado, si se desea inicializar algún atributo de la clase por medio del constructor, debe contener al inicio el parámetro \textit{self}, como vemos enseguida:
\begin{lstlisting}
    # Declara clase.
    class Perro:
        # Constructor.
        def __init__(self, nombre, raza):
            # Atributos: nombre y raza.
            self.nombre = nombre
            self.raza = raza

        # Método que despliega sonido del perro.
        def sonido(self):
            print("Woof")

    # Declara objeto de la clase Perro.
    perrito = Perro("chuchis", "amarillo")
    # Despliega el nombre del perro.
    print(perrito.name)
    # Llamada al método sonido de la clase Perro.
    perrito.sonido()
      
    # Despliega:
    # chuchis
    # Woof
\end{lstlisting}


\subsection{Destructores}

Este es el método llamado \textbf{\_\_ del\_\_}, el cual es llamado cuando un objeto de una clase deja de ser utilizado, es eliminado o la ejecución del programa ha finalizado.
\begin{lstlisting}
    def __del__(self):
\end{lstlisting}


\subsection{Herencia}

Para heredar los atributos y métodos de una clase, primero debemos escribir el nombre de la clase heredera, seguido de paréntesis, en su contenido, escribimos el nombre de la clase a heredar y dos puntos:
\begin{lstlisting}
    nombreClase(claseHeredera):
\end{lstlisting}

Con esto, pasamos el contenido de una clase a otra, además, sabemos que la clase heredera es una \textbf{superclase} o \textbf{clase padre}, mientras que una heredada es una \textbf{sub clase} o \textbf{clase hija}, en la clase hija podremos crear métodos independientes a la clase padre. Veamos el siguiente ejemplo.
\begin{lstlisting}
    # Declara clase base.
    class Animal:
        # Constructor.
        def __init__(self, nombre, raza):
            self.nombre = nombre
            self.raza = raza

        # Método de clase base.
        def despliegue(self):
            print("Mi nombre es" ++ self.nombre ++ " y mi raza es " ++ self.raza)

    # Declara clase hija de Animal.
    class Perro(Animal):
        # Método de clase hija.
        def sonido(self):
            print("Woof")

    # Declara objeto de clase hija.
    perrito = Perro("chuchis", "chihuahua")
    # Llamada a métodos de clase hija,
    perrito.despliegue()
    perrito.sonido()
\end{lstlisting}

Para acceder a los métodos de una superclase utilizamos la palabra reservada \textbf{super()}, y para acceder a los atributos de la superclase, utilizamos la palabra reservada \textbf{self}.
\begin{lstlisting}
    # Clase base.
    class A:
        # Constructor con atributo de prueba.
        def __init__(self, prueba):
            self.prueba = prueba
        # Método propio.
        def spam(self):
            print(1)

    # Clase derivada de A.
    class B(A):
        # Método que utiliza el método y atributo de su clase base.
        def spam(self):
            print(2)
            super().spam()
            print(self.prueba)

    # Declara objeto de la clase B y llama directamente a su método.
    B("hola mundo").spam()
\end{lstlisting}


\subsection{Abstraccion}

De manera rápida y directa, Python no posee palabras reservadas para indicar que una clase y sus métodos son abstractos, para acabar con esta problemática, debemos importar la \textbf{clase ABC} de la \textbf{librería abc} y su \textbf{decorador abstractmethod}:
\begin{lstlisting}
    from abc import ABC, abstractmethod
\end{lstlisting}

La \textit{clase abc} debe ser la clase padre de la clase que deseamos que sea abstracta, esto permite, más que heredemos atributos o métodos, que utilicemos el decorador que recién importamos en los métodos de la clase derivada, podemos pensar en la clase abc meramente como una etiqueta para la clase a convertir en abstracta.

Si recordamos, una clase abstracta no puede tener objetos, y todos sus métodos deben ser utilizados en las clases hijas, veamos la creación de una clase abstracta en el siguiente ejemplo:
\begin{lstlisting}
    # Importa clase ABC y decorador abstractmethod de la librería abc.
    from abc import ABC, abstractmethod

    # Declara clase derivada abstracta de clase base ABC.
    class Figuras(ABC):
        # Declara métodos abstractos sin inicializar.
        @abstractmethod
        # Regresa el volumen de la figura.
        def calcular_volumen(self):
            pass

        @abstractmethod
        # Despliega el volumen y medidas para desplegar el volumen.
        def desplegar_volumen(self):
            pass
\end{lstlisting}

El cuerpo del método abstracto no debe quedar vacío, por lo que se utiliza la palabra reservada \textbf{pass}. Ahora crearemos una clase derivada de la clase abstracta:
\begin{lstlisting}
    class Cilindro(Figuras):
        def __init__(self, radio, altura):
            self.radio = radio
            self.altura = altura

        def calcular_volumen(self):
            return math.pi * math.pow(self.radio, 2) * self.altura

        def desplegar_volumen(self):
            print("""
            ---Datos del cilindro---
            La medida del radio del cilindro es de {}u.
            La medida de la altura del cilindro es de {}u.
            El volumen del cilindro es de {}u^3.
            """.format(self.radio, self.altura, 										round(self.calcular_volumen(), 3)))
\end{lstlisting}

Si quisiéramos crear más clases derivadas que tuvieran un comportamiento distinto lo podríamos hacer (un cono, un cilindro, un cubo, etc), siempre y cuando declaremos los métodos abstractos en cada clase derivada.

Declaremos un objeto de la clase Cilindro:
\begin{lstlisting}
    # Declara objeto Cilindro y se le pasan valores para el radio
    # y la altura.
    cilindro = Cilindro(5.5, 10.66)
    # Llamada al método que despliega el volumen del cilindro.
    cilindro.desplegar_volumen()
\end{lstlisting}

En caso de querer declara un objeto de la clase Figuras, el lenguaje lanzará un error.


\subsection{Polimorfismo}

No es más que el pensamiento de que una función puede adoptar un comportamiento distinto en las distintas clases derivadas de una base, como lo que se acaba de ver en el punto anterior, pero sin ser abstracto, pondremos un breve ejemplo sobre lo que es polimorfismo:
\begin{lstlisting}
    # Clase base.
    class Animal:
        # Destructor.
        def __del__(self):
            print("Objeto Animal Eliminado.")

        # Método que adoptará distinto comportamiento en sus clases derivadas (polimorfismo).
        def desplazarse(self):
            pass

    # Clase derivada de clase base Animal.
    class Ave(Animal):
        # Constructor.
        def __init__(self, nombre):
            self.nombre = nombre

        # Destructor.
        def __del__(self):
            print("Objeto Ave Eliminado.")

        # Método que adopta su propio comportamiento al de la clase base.
        def desplazarse(self):
            print("Hola, mi nombre es {} y me desplazo volando.".format(self.nombre))

    class Mamifero(Animal):
        def __init__(self, nombre):
            self.nombre = nombre

        def __del__(self):
            print("Objeto Mamifero Eliminado.")

        def desplazarse(self):
            print("Hola, mi nombre es {} y me desplazo caminando.".format(self.nombre))

    # Declara objetos de clase Ave y Mamifero.
    ave = Ave("Mauricio")
    mamifero = Mamifero("Gonzalo")
    # Llama a los métodos de desplazarse.
    ave.desplazarse()
    mamifero.desplazarse()
    
    # Imprime:
    # Hola, mi nombre es Mauricio y me desplazo volando.
    # Hola, mi nombre es Gonzalo y me desplazo caminando.
\end{lstlisting}

La clase base posee un método que tendrá un comportamiento distinto en sus clases derivadas, vemos que aquí también aparece la palabra reservada \textbf{pass} para no dejar completamente vacío el cuerpo de esta función. A diferencia de la abstracción, si declaramos un objeto de la clase base, el lenguaje no lanzará un error, si lo vemos como jerarquía, primero está el polimorfismo y luego la abstracción, pero ambos son dos conceptos muy ligados a la programación orientada a objetos.


\subsection{Métodos mágicos}

Son aquellos que tienen dos guiones bajos al inicio y final del nombre, funcionan para representar funcionalidades que no pueden ser creadas con un método regular, por ejemplo, poder realizar operaciones aritméticas con objetos de clases, de hecho, los constructores (\_\_ init\_\_) es un método mágico. La lista de métodos mágicos más comunes es:
\begin{itemize}
	\item \textbf{\_\_init\_\_} para constructores.
	\item \textbf{\_\_add\_\_} para sumas.
	\item \textbf{\_\_sub\_\_} para restas.
	\item \textbf{\_\_mul\_\_} para multiplicaciones.
	\item \textbf{\_\_truediv\_\_} para divisiones regulares.
	\item \textbf{\_\_floordiv\_\_} para divisiones enteras.
	\item \textbf{\_\_mod\_\_} para residuos.
	\item \textbf{\_\_pow\_\_} para potencias.
	\item \textbf{\_\_and\_\_} para comparaciones y.
	\item \textbf{\_\_xor\_\_} para comparaciones xor.
	\item \textbf{\_\_or\_\_} para comparaciones o.
	\item \textbf{\_\_lt\_\_} para comparación menor.
	\item \textbf{\_\_le\_\_} para comparación menor igual.
	\item \textbf{\_\_eq\_\_} para comparación igual.
	\item \textbf{\_\_ne\_\_} para comparación distinto.
	\item \textbf{\_\_gt\_\_} para comparación mayor.
	\item \textbf{\_\_ge\_\_} para comparación mayor igual.
\end{itemize}


\subsection{Ocultación de datos}

La \textit{encapsulación} de atributos y métodos de una clase es algo normal en la POO y otros lenguajes de programación, sin embargo, este concepto no existe como tal en Python, podemos utilizar la \textbf{ocultación de datos} para evitar que secciones de código fuera de la clase accedan a los estos miembros privados, para complementar, solamente existen los "métodos  privados", no existen los métodos protegidos. Un método puede hacerse privado poniendo un guión bajo antes del nombre, veamos este ejemplo.
\begin{lstlisting}
    # Declara clase.
    class Cola:
        #Construtor con lista "escondido" privada.
        def __init__(self, contenido):
            self._escondido = list(contenido)

    # Método que inserta valor en la cola.
        def push(self, valor):
            self._escondido.insert(0, valor)

    # Declara objeto Cola.
    cola = Cola([1, 2, 3])
    print(cola)
    cola.push(0)
    print(cola)
    print(cola.escondido)
      
    # Imprime:
    # Queue[1, 2, 3].
    # Queue[0, 1, 2, 3].
    # [0, 1, 2, 3].
\end{lstlisting}

Al no existir la encapsulación en Python, el guión bajo en métodos es solo una convención, ya que si podemos acceder a estos métodos por fuera de la clase. Si seguimos buscando utilizar miembros privados, agregamos un guión más al inicio del nombre del miembro, con esto logramos un efecto mayor de privacidad, pero aún podemos acceder a este método por fuera, solamente tenemos que llamarlo de una forma particular.
\begin{lstlisting}
    class Ejemplo:
        # Declara atributo privado.
        __huevo = 7
        # Método que imprime el valor del atributo privado.
        def imprimir_Huevo(self)
            print(self.__huevo)
         
    # Declara objeto Ejemplo.
    e = Ejemplo()
    # Llama al método de la clase.
    s.imprimir_Huevo()
    # Imprime el valor del atributo privado.
    print(e._Ejemplo__huevo)
\end{lstlisting}

Es entonces que, para acceder al miembro privado fuera de la clase, debemos escribir el nombre del objeto, un punto y guión bajo, el nombre de la clase y enseguida el nombre del atributo:
\begin{lstlisting}
    e._Ejemplo__ Variable
\end{lstlisting}


\subsection{Métodos estáticos y de clase}

Los \textbf{métodos de clase} son como una clase dentro de otra clase, o un tipo especial de método dentro de clase que no puede ser llamado por medio de un objeto de clase, sino que son llamados directamente desde el nombre de la clase, como las clases no pueden ser utilizadas directamente sin un objeto de por medio, para lograr utilizar los métodos de clases se debe utilizar una variable de por medio, que termina actuando como un objeto del método de clase.

Para diferenciar estos métodos de los otros, utilizamos el decorador \textbf{classmethod}, como en el siguiente ejemplo.
\begin{lstlisting}
    # Declara clase.
    class Rectangulo:
        # Constructor
        def __init__(self, ancho, alto):
            self.ancho = ancho
            self.alto = alto

        # Método que calcula área del rectángulo.
        def area(self):
            return self.ancho * self.alto

        # Método de clase que declara un cuadrado.
        @classmethod
        # Así como self, el método de clase debe tener su palabra reservada,
        # en este caso, utilizamos la palabra "cls".
        def cuadrado(cls, lado)
            return cuadrado(lado, lado)

    # Variable que recibe el método de la clase Rectangulo.
    sqr = Rectangulo.cuadrado(5)
    # Llama al método que calcula el área del rectángulo, pero
    # aplicado a un cuadrado.
    print(sqr.area())
\end{lstlisting}

Los métodos de clase deben tener su propia palabra reservada, así como las clases tienen la palabra \textbf{self}, estas dos palabras pueden tener el nombre que desees, sin embargo, \textbf{self} y \textbf{cls} son conocidas en muchos sitios y utilizadas en muchas partes, por lo que es recomendable utilizarlas para no confundir a quien lea tu código.

Los \textbf{métodos estáticos} son iguales a los métodos regulares de clases o fuera de clases, pero no requieren de la palabra reservada \textit{self} dentro de sus parámetros, y puede llamarse a estos métodos por medio de la clase sin necesidad de un objeto, utilizamos el decorador \textbf{staticmethod}, veamos el ejemplo.
\begin{lstlisting}
    # Declara clase.
    class Shape:
        # Constructor.
        def __init__(self, ancho, alto):
            self.ancho = ancho
            self.alto = alto

    # Método estático.
    @staticmethod
        def area(ancho, alto):
            return ancho * alto

    # Variables reciben valores
    w = int(input())
    h = int(input())
    # Imprime el área de la figura, usando el metodo estático sin necesidad
    # de crear un objeto de la clase.
    print(Shape.area(w, h))
\end{lstlisting}


\subsection{Propiedades}

El decorador \textbf{property} ayuda a poder acceder a los métodos de una clase como si fueran atributos, o acceder a atributos de la clase por medio de un método, el uso más común de este decorador es volver atributos a solo-lectura. Otro tipo de decorador propiedad son los \textbf{Setters} y \textbf{Getters}, el primero asigna un valor a un atributo, y el segundo obtiene dicho valor, por ejemplo.
\begin{lstlisting}
    # Declara clase.
    class Trabajador:
        # Constructor
        def __init__(self, primero, segundo):
            self.primero = primero
            self.segundo = segundo

        # Método que define un correo electrónico, es tratado como atributo.
        @property
        def correo(self):
            return "{}.{}@email.com".format(self.primero, self.segundo)

        # Método que define el nombre completo, es tratado como atributo.
        @property
        def nombre_completo(self):
            return "{} {}".format(self.primero, self.segundo)

        # Método con decorador Setter que define el nombre completo
        # del trabajador.
        @nombre_completo.setter
        def nombre_completo(self, nombre):
            primero, ultimo = nombre.split(" ")
            self.primero = primero
            self.segundo = segundo

    # Declara objeto Trabajador.
    emp1 = Trabajador("Luis", "Miguel")
   
    # Asigna nombre completo al objeto, cambiando los atributos de dicho objeto.
    emp1 = nombre_completo("Ernesto Casas")

    # Imprime el primer nombre.
    print(emp1.primero)
    # Imprime el segundo nombre.
    print(emp1.segundo)
    # Llama al metodo nombre_completo, pero al ser una propiedad, no requiere de parentesis.
    print(emp1.nombre_completo)
\end{lstlisting}

Vemos que se crea un objeto de la clase Trabajador con ciertos valores, después se cambia el nombre del trabajador por medio de un Setter, logrando cambiar los valores de los atributos del objeto, el método \textit{nombre\_ completo} regresa los atributos \textit{primero} y \textit{segundo} como una sola cadena, con el decorador property este método es tratado como un atributo más, por lo que descartamos el uso de paréntesis al final del nombre, lo mismo aplica con el método \textit{correo}.



\section{Excepciones}

El manejo de excepciones se logra utilizando el bloque \textbf{try:-except:}, donde dentro del bloque try se escribe el código que puede fallar, y el bloque exception contiene el código en caso de que el error se de. Algunas de las excepciones por defecto de Python son:
\begin{itemize}
	\item \textbf{ImportError}: cuando una importación de biblioteca falla.
	\item \textbf{IndexError}: cuando se intenta acceder a un índice inexistente de una colección de datos.
	\item \textbf{NameError}: cuando se trata de usar una variable desconocida.
	\item \textbf{SyntaxError}: cuando el código no se puede analizar correctamente.
	\item \textbf{TypeError}: cuando un tipo de dato no corresponde con la función o variable.
	\item \textbf{ValueError}: cuando un valor no corresponde con el tipo de dato de la función o variable.
	\item En caso de que no se asigne un tipo de excepción a bloque except, quiere decir que except va a atrapar todos los tipos de excepciones existentes.
\end{itemize}

Un bloque except puede contener más de una excepción, contenida en paréntesis, o tener múltiples bloques except para cada tipo de excepción.
\begin{lstlisting}
    # Bloque try-except.
    try:
        variable = 10
        # Suma de un entero y una cadena.
        print(variable + "hello")
        print(variable / 2)
    # Excepción por si se divide entre 0.
    except ZeroDivisionError:
        print("Divided by zero")
    # Excepción por si un valor o tipo no coincide.
    except (ValueError, TypeError):
        print("Error occurred")
\end{lstlisting}


\subsection{finally y else}

El bloque \textbf{finally} se ejecuta después del bloque try, sin importar si el bloque except se ejecutó o no, además, podemos utilizar el bloque \textit{else}, este se ejecuta únicamente si el bloque try se ejecutó con normalidad, es decir, sin error alguno.
\begin{lstlisting}
    # Bloque try-except.   
    try:
        # Variable que recibe un entero.
        val = int(input())
        print(val)
    # Excepción por si el tipo ingresado no es válido.
    except TypeError:
        print("Ese no es un numero")
    # Si no hubo error, se despliega mensaje.
    else:
        print("Todo salio bien")
    # Bloque con mensaje final.
    finally:
        print("Hemos terminado")
\end{lstlisting}


\subsection{Lanzando excepciones}

Podemos lanzar o llamar excepciones si alguna condición o se cumple, con la palabra reservada \textbf{raise}, seguido del nombre de la excepción a llamar.
\begin{lstlisting}
    # Asignación de valor a variable.   
    tweet = input()

    # Bloque try-except-else.
    try:
        #Si el largo de la variable es mayor a 42, lanza excepción.
        if len(tweet) > 42:
            raise ValueError
    except:
        print("Error")
    else:
        print("Posted")
\end{lstlisting}



\section{Trabajando con archivos}

Para abrir un archivo en Python, utilizamos la función \textbf{open}, y dentro de sus paréntesis, va la ruta del archivo (si el archivo está en la carpeta actual donde se ubica la solución, solamente es necesario escribir el nombre, si está en otra carpeta, se indica la ruta completa del archivo).
\begin{lstlisting}
    miArchivo = open("prueba.txt")
\end{lstlisting}

Podemos especificarle el \textbf{como} se trabajará el archivo, es decir, si va a ser sobreescrito, solo leído, o escribir un archivo binario.
\begin{lstlisting}
    # Modo Escritura.
    open("prueba.txt", "w")
    # Modo Lectura.
    open("prueba.txt", "r")
    # Modo Escritura Binaria.
    open("prueba.txt", "wb")
    # Modo Lectura Binaria.
    open("prueba.txt", "rb")
\end{lstlisting}

Una vez abrimos y trabajamos el archivo, debemos cerrarlo, esto lo conseguimos con la función \textbf{close()}.
\begin{lstlisting}
    miArchivo = open("prueba.txt")
    miArchivo.close()
\end{lstlisting}

Es una buena practica siempre abrir, trabajar el archivo, y cerrarlo, para no desperdiciar recursos, podemos evitar esto utilizando un bloque try-finally o la instrucción \textbf{with}, donde se crea una variable temporal y se utiliza únicamente dentro del bloque identado, después de eso, la variable es borrada.
\begin{lstlisting}
    # Opción 1:
    # Declara variable f con instrucción with abriendo un archivo.
    with open("archivo.txt") as f:
        # Imprime el contenido del archivo.
        print(f.read())
      
    # Se sale del bloque identado, la variable f es borrada.
    # Opción 2:
    # Bloque try-finally para abrir y cerrar un archivo.
    try:
        # Variable que declara un archivo.
        f = open("archivo.txt")
        # Imprime el contenido del archivo.
        print(f.read())
    finally:
        # Después del bloque try, se ejecuta finally que cierra el archivo.
        f.close()
\end{lstlisting}


\subsection{Leyendo archivos}

Con la función \textbf{read()} del archivo que hemos abierto podemos desplegar el contenido del mismo.
\begin{lstlisting}
    # Declara variable con un archivo.
    archivo = open("libros.txt")
    # Asigna a una variable el contenido del archivo.
    aux = archivo.read()
    # Imprime el contenido del archivo.
    print(aux)
    # Cierra el archivo.
    archivo.close()
\end{lstlisting}

En caso de que queramos leer únicamente cierta cantidad de caracteres del archivo, podemos pasarle como parámetro a la función \textit{read()} la cantidad de caracteres deseados, la función regresará la cantidad deseada.
\begin{lstlisting}
    # Declara variable con un archivo.
    archivo = open("libros.txt")
    # Asigna a una variable cierta cantidad de caracteres del archivo.
    aux = archivo.read(7)
    aux2 = archivo.read(5)
    # Imprime 7 y 5 caracteres del archivo.
    print(aux)
    print(aux2)
    # Cierra el archivo.
    archivo.close()
\end{lstlisting}

Otra forma de imprimir el contenido de un archivo es por medio de la función \textbf{readlines()}, el cual regresa una lista con las línea por línea el contenido de archivo, debemos utilizar un ciclo \textit{for} para poder imprimir línea a línea. A su vez, si no queremos utilizar las dos funciones anteriormente mencionadas, podemos simplemente juntar un ciclo for y el puro nombre del archivo, como vemos enseguida.
\begin{lstlisting}
    # Declara variable con un archivo.
    file = open("filename.txt", "r")
    # Ciclo for que lee línea a línea el contenido de un archivo.
    for line in file.readlines():
        #Imprime la linea.
        print(line)
    # Cierra el archivo.
    file.close()

    # Declara variable con un archivo.
    file = open("filename.txt", "r")
    # Ciclo for que imprime el contenido del archivo.
    for line in file:
        # Imprime la línea.
        print(line)
    # Cierra el archivo.
    file.close()
\end{lstlisting}

La diferencia entre \textit{read()}, \textit{readlines()} y \textit{in file}, es que estos dos últimos agregan un salto de línea adicional al que da \textit{print()} cuando despliega algo.


\subsection{Escribiendo en archivos}

Con la función \textbf{write()} escribimos contenido en el archivo, si el \textbf{modo} del archivo es \textit{write} y el archivo ya existe, el archivo es sobreescrito, si no existe, Python lo crea en la ruta establecida.

Recordemos que el \textit{modo a} permite agregar contenido a un archivo ya existente, así que podemos abrir un archivo con dicho modo y Python te posicionará en la última línea del archivo para que puedas agregar nuevo contenido.
\begin{lstlisting}
    # Declara variable con un archivo en modo write.
    file1 = open("filename.txt", "w")
    # Sobre escribe el contenido del archivo si ya existe, o lo crea y escribe el contenido en el.
    file.write("Borron y cuenta nueva")
    # Cierra el archivo.
    file.close()

    # Declara variable con un archivo en modo append.
    file2 = open("filename.txt", "r")
    # Sobre escribe el contenido del archivo en la última línea.
    file.write("\nAgregado al final")
    # Cierra el archivo.
    file.close()
\end{lstlisting}

La función write regresa el número de caracteres escritos en el archivo, si es que se pudieron escribir correctamente.
\begin{lstlisting}
    # Declara variable con un archivo en modo write.
    file1 = open("filename.txt", "w")
    # Asigna a una variable el número de caracteres escritos en el archivo.
    num_caracteres = file1.write("abcde")
    # Despliega 5.
    print(num_caracteres)
    # Cierra el archivo.
    file.close()
\end{lstlisting}


\subsection{Comprobar la existencia de un archivo}

Para saber si existe un archivo o directorio dentro de una carpeta en Windows con Python requerimos importar el módulo \textbf{os} para llamadas al sistema:
\begin{lstlisting}
    import os
\end{lstlisting}

Sea cual sea la situación o problema que estemos programando o resolviendo, podemos utilizar la siguiente función que comprueba la existencia de un archivo o directorio en nuestro sistema de archivos:
\begin{lstlisting}
    import os

    def verificar_archivo():
        if os.path.exist(ruta):
   	    # Instrucciones si la ruta o archivo existe.
        else:
            # Instrucciones si la ruta o archivo no existe.
\end{lstlisting}


\subsection{Obtener una lista con todos los archivos de una carpeta}

La siguiente función resulta útil si queremos seleccionar un archivo de una carpeta, pero primero queremos ver todo el contenido de la carpeta donde se encuentra almacenado:
\begin{lstlisting}
    import os

    def lista_archivos_en_directorios():
        return [arch.name for arch in os.scandir(ruta) if arch.is_file()]
\end{lstlisting}

\textit{Nota}: esta función también requiere del módulo \textbf{os} para funcionar.


\subsection{Eliminar un archivo del sistema}

Requiere importar la clase \textbf{remove} del módulo \textbf{os} para funcionar:
\begin{lstlisting}
    from os import remove
\end{lstlisting}

Entonces, si en la carpeta "trabajos" tengo tres archivos: "final.docx", "ahora si final.docx" y "final final final en serio.docx" y quiero borrar el último, basta con utilizar la siguiente instrucción para eliminarlo:
\begin{lstlisting}
    from os import remove

    remove("trabajos/final final final en serio.docx")
\end{lstlisting}

Tenga en cuenta cómo funcionan las rutas en el sistema operativo donde esté programando.


% Final del documento.
\end{document}